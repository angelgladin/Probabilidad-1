%%%%%
    %
    % Copyright (C) 2019 Ángel Iván Gladín García
    %
    % This program is free software: you can redistribute it and/or modify
    % it under the terms of the GNU General Public License as published by
    % the Free Software Foundation, either version 3 of the License, or
    % (at your option) any later version.
    %
    % This program is distributed in the hope that it will be useful,
    % but WITHOUT ANY WARRANTY; without even the implied warranty of
    % MERCHANTABILITY or FITNESS FOR A PARTICULAR PURPOSE.  See the
    % GNU General Public License for more details.
    %
    % You should have received a copy of the GNU General Public License
    % along with this program.  If not, see <http://www.gnu.org/licenses/>.
%%%%%

%%%%%%%%%%%%%%%%%%%%%%%%%%%%%%%%%%%%%%%%%%%%%%%%%%%%%%%%%%%%%%%%%%%%%%%%%%%%%%%%%%%%%%%%%
\documentclass[11pt,letterpaper]{report}
\usepackage[margin=.7in]{geometry}
\usepackage[utf8]{inputenc}
\usepackage[spanish]{babel}
\decimalpoint

\usepackage{listings}
\usepackage{color}
\usepackage{fontawesome}
\usepackage{graphicx}
\usepackage{enumerate}
\usepackage{enumitem}
\usepackage{float}

\usepackage{longtable}
\usepackage{hyperref}
\usepackage{commath}

\usepackage{bbm}
\usepackage{cancel}
\usepackage{dsfont}
\usepackage{mathrsfs}
\usepackage{amsmath,amsthm,amssymb}
\usepackage{mathtools}
\usepackage{longtable}

\usepackage{tikz}
\usetikzlibrary{trees}
\usepackage{verbatim}

% Set the overall layout of the tree
\tikzstyle{level 1}=[level distance=3.5cm, sibling distance=3.5cm]
\tikzstyle{level 2}=[level distance=3.5cm, sibling distance=2cm]

% Define styles for bags and leafs
\tikzstyle{bag} = [text width=4em, text centered]
\tikzstyle{end} = [circle, minimum width=3pt,fill, inner sep=0pt]
%%%%%%%%%%%%%%%%%%%%%%%%%%%%%%%%%%%%%%%%%%%%%%%%%%%%%%%%%%%%%%%%%%%%%%%%%%%%%%%%%%%%%%%%%%%%%%%%
\DeclarePairedDelimiter\floor{\lfloor}{\rfloor}
\usepackage{import}

\usepackage[utf8]{inputenc}

\usepackage{listings}
\usepackage{color}

\definecolor{codegreen}{rgb}{0,0.6,0}
\definecolor{codegray}{rgb}{0.5,0.5,0.5}
\definecolor{codepurple}{rgb}{0.58,0,0.82}
\definecolor{backcolour}{rgb}{0.95,0.95,0.92}

\lstdefinestyle{mystyle}{
    backgroundcolor=\color{backcolour},   
    commentstyle=\color{codegreen},
    keywordstyle=\color{magenta},
    numberstyle=\tiny\color{codegray},
    stringstyle=\color{codepurple},
    basicstyle=\footnotesize,
    breakatwhitespace=false,         
    breaklines=true,                 
    captionpos=b,                    
    keepspaces=true,                 
    numbers=left,                    
    numbersep=5pt,                  
    showspaces=false,                
    showstringspaces=false,
    showtabs=false,                  
    tabsize=2
}

\lstset{style=mystyle}
%%%%%%%%%%%%%%%%%%%%%%%%%%%%%%%%%%%%%%%%%%%%%%%%%%%%%%%%%%%%%%%%%%%%%%%%%%%%%%%%%%%%%%%%%


%%%%%%%%%%%%%%%%%%%%%%%%%%%%%%%%%%%%%%%%%%%%%%%%%%%%%%%%%%%%%%%%%%%%%%%%%%%%%%%%%%%%%%%%%
\newcommand{\Z}{\mathbb{Z}}
\newcommand{\N}{\mathbb{N}}
\newcommand{\Q}{\mathbb{Q}}
\newcommand{\R}{\mathbb{R}}
\newcommand{\Pro}{\mathds{P}}
\newcommand{\Oh}{\mathcal{O}} %% Notacion "O"
\newcommand{\lra}{\longrightarrow}
\newcommand{\ra}{\rightarrow}
\newcommand{\ord}{\text{ord}}
\newcommand{\sol}{\textbf{\underline{Solución}: }} %% Solucion
\newcommand{\af}{\textbf{\underline{Afirmación}: }}
\newcommand{\cej}{\textbf{\underline{Contraejemplo}: }}

%%%%%%%%%%%%%%%%%%%%%%%%%%%%%%%%%%%%%%%%%%%%%%%%%%%%%%%%%%%%%%%%%%%%%%%%%%%%%%%%%%%%%%%%%

\begin{document}

%%%%%%%%%%%%%%%%%%%%%%%%%%%%%%%%%%%%%%%%%%%%%%%%%%%%%%%%%%%%%%%%%%%%%%%%%%%%%%%%%%%%%%%%%
\title{
        Universidad Nacional Autónoma de México\\
        Facultad de Ciencias\\
        Probabilidad I\\
    \vspace{1cm}
    \large
        \textbf{Tarea 5}\\
        \textbf{Variables Aleatorias Discretas II}
}
\author{
    Ángel Iván Gladín García\\
    No. cuenta: 313112470\\
    \texttt{angelgladin@ciencias.unam.mx}
}
\date{9 de noviembre 2019}
\maketitle
%%%%%%%%%%%%%%%%%%%%%%%%%%%%%%%%%%%%%%%%%%%%%%%%%%%%%%%%%%%%%%%%%%%%%%%%%%%%%%%%%%%%%%%%%

%%%%%%%%%%%%%%%%%%%%%%%%%%%%%%%%%%%%%%%%%%%%%%%%%%%%%%%%%%%%%%%%%%%%%%%%%%%%%%%%%%%%%%%%%
\newtheorem{theorem}{Teorema}
\newtheorem{example}{Ejemplo}
\newtheorem{corollary}{Corolario}
\newtheorem{lemma}{Lemma}
\newtheorem{definition}{Definicion}
\newtheorem{prop}{Proposicion}
%%%%%%%%%%%%%%%%%%%%%%%%%%%%%%%%%%%%%%%%%%%%%%%%%%%%%%%%%%%%%%%%%%%%%%%%%%%%%%%%%%%%%%%%%

%%%%%%%%%%%%%%%%%%%%%%%%%%%%%%%%%%%%%%%%%%%%%%%%%%%%%%%%%%%%%%%%%%%%%%%%%%%%%%%%%%%%%%%%%
\begin{enumerate}

%%%%%%%%%%%%%%% (1)
\item Si cada persona en una comunidad de 1000 habitantes tiene un 1\% de chances de resultar
infectado de un peligroso virus. Suponiendo que una persona resulta infectada independientemente de
lo que ocurra con otra persona de la comunidad. Calcula la probabilidad de que:

\sol Sea $X$ la variable aleatoria que representa el número de gente infectada.

Recordando que la f.m.p de una variable aleatoria que se destribuye binomial, ósea $X \sim Bin(n,p)$ con
parámetros $(n,p)$ está dada por:
\[
    p(i) = \binom{n}{i} p^n (1-p)^{n-i} \qquad i = 0,1,\ldots,n
\]


\begin{itemize}
    \item Exactamente haya 10 personas infectadas
    
    \sol
    $$P(X=10) = \binom{1000}{10} (0.01)^10 (0.99)^{990} \approx 0.12574$$
    
    \item A lo mas haya 16 personas infectadas
    
    \sol
    $$P(X \leq 16) = \sum_{i=0}^{16} \binom{1000}{i} (0.01)^i (0.99)^{100-i} \approx 0.9736$$

    \item Haya entre 12 y 14 personas infectadas
    
    \sol
    $$P(12 \leq X \leq 14) = \sum_{i=12}^{14} \binom{1000}{i} (0.01)^i (0.99)^{100-i} \approx 0.22023$$

    \item Alguien resulte infectado
    
    \sol
    $$P(X \geq 1) = 1 - P(X = 0) = 1 - \binom{1000}{0} = 1 - 0.99^{1000} \approx 0.9999$$

\end{itemize}

%%%%%%%%%%%%%%% (2)
\item En 1693 Samuel Pepys le escribió una carta a Isaac Newton en donde le planteo el siguiente
problema: ¿Cuál de los siguientes tres planteamientos es mas probable que ocurra?
\begin{itemize}
    
    \item Se lanzan 6 dados y aparece al menos un 6's
    
    \sol La probabilidad de que aparezca un 6 en un dado es $\frac{1}{6}$ pero si obtenemos el complemento
    que es $1-\frac{1}{6} = \frac{5}{6}$ que es que no aparezca un 6. Hacemos esto porque nos pide
    calcular \textit{al menos} $6n$ veces (esto nos ayudará para los demás ejercicios).

    Entonces denotamos a $X$ v.a. como el número de 6's que aparecen en $6n$ dados, entonces\\
    $X \sim Bin(n, \frac{1}{6})$. Teniendo una versión generalizada como,
    \[
        P(X=x) = 1 - \binom{6n}{x} (\frac{1}{6})^{x} (\frac{5}{6})^{6n-x}
    \]
    
    Entonces si lanzamos un dado 6 veces, ósea $X \sim Bin(1, \frac{1}{6})$
    \[
        P(X \geq 1) = 1 - P(X = 0) = 1 - \binom{6}{0} (\frac{1}{6})^0 (\frac{5}{6})^6  \approx 0.6651
    \]

    \item Se lanzan 12 dados y aparecen al menos dos 6’s
    
    \sol $X \sim Bin(2, \frac{1}{6})$

    \[
        P(X \geq 2) = 1 - \sum_{x=0}^{1} P(X = x)  \approx 0.6187
    \]
    
    \item Se lanzan 18 dados y aparecen al menos tres 6’s
    
    \sol $X \sim Bin(3, \frac{1}{6})$
    \[
        P(X \geq 2) = 1 - \sum_{x=0}^{2} P(X = x) \approx 0.5973
    \]
\end{itemize}
    
Como dato adicional Sir Isaac Newton ademas de ser uno de los mas grandes científicos de la historia,
durante un gran lapso de su vida fue un funcionario publico, manejo digamos el SAT de lo que hoy es
el Reino Unido. Y estaba a favor usar la horca o quemar gente si evadian impuestos, eran buenos tiempos.

%%%%%%%%%%%%%%% (3)
\item \faFrownO

%%%%%%%%%%%%%%% (4)
\item Un viejo juego de dados llamado \emph{Chuck a Luck} popular en el EE.UU. del S XIX. Consiste en que
un jugador apuesta a algún número entre el 1 y el 6 entonces se lanzan 3 dados. Si el número al que le aposto el jugador
aparece una, dos o tres veces al lanzar los dados el jugador recibe una, dos o tres veces lo que aposto, mas su apuesta
inicial y pierde su apuesta inicial en cualquier otro caso. Sea $X$ la variable aleatoria que denota: La ganancia del
jugador por unidad del principal apostado inicialmente.

\begin{itemize}
    \item Obtén la función de masa de probabilidad asociada a $X$.
    
    \sol Asumiendo que el dado es \emph{justo} y cada evento es independiente del otro, entonces el número de veces que
    el número de apuesta que aparece es una \textbf{variable aleatoria binomial} con parámetros $(3, \frac{1}{6})$.
    Por lo tanto la f.m.p. es: (evaluando con los valores).
    $$p(i) = \binom{3}{i}p^i(1-\frac{1}{6})^{3-i}$$


    \item Calcula el Valor esperado de la ganancia del jugador por unidad apostada.
    
    \sol
    Obteniendo la probabilidad de que pierda una unidad y que gane 1,2 y unidades,
    \begin{align*}
        P(X=-1) = \binom{3}{0} (\frac{1}{6})^0 (\frac{5}{6})^3 = \frac{125}{216}\\
        P(X=1) = \binom{3}{1} (\frac{1}{6})^1 (\frac{5}{6})^2 = \frac{75}{216}\\
        P(X=2) = \binom{3}{2} (\frac{1}{6})^2 (\frac{5}{6})^1 = \frac{15}{216}\\
        P(X=3) = \binom{3}{3} (\frac{1}{6})^3 (\frac{5}{6})^0 = \frac{1}{216}\\
    \end{align*}
    
    Obteniendo el valor esperado,
    \begin{align*}
        \mathds{E}(X)
            &= -1 \cdot \frac{125}{216} + 1 \cdot \frac{75}{216} + 2 \cdot \frac{15}{216}
                + 3 \cdot \frac{1}{216}\\
            &= \frac{-17}{216}
    \end{align*}
\end{itemize}

%%%%%%%%%%%%%%% (5)
\item Si $X \sim Bin(n, p)$.

Prueba que $\mathds{E}(\frac{1}{X+1}) = \frac{1-(1-p)^{n+1}}{(n+1)p}$

\begin{proof}
\begin{align*}
    \mathds{E}(\frac{1}{X+1})
        &= \sum_{k=0}^{n} \frac{1}{k+1} f(k) && \text{Def. esperanza}\\
        &= \sum_{k=0}^{n} \frac{1}{k+1} \binom{n}{k} p^k (1-p)^{n-k} && \text{Por hip. $X \sim Bin$}\\
        &= \sum_{k=0}^{n} \frac{1}{k+1} \frac{n!}{k!(n-k)!} p^k (1-p)^{n-k} && \text{Reescribiendo coef. binomial}\\
        &= \sum_{k=0}^{n} \frac{n!}{(k+1)!(n-k)!} p^k (1-p)^{n-k} && \text{Álgebra}\\
        &= \frac{1}{(n+1)p}\sum_{k=0}^{n} \frac{(n+1)!}{(k+1)!(n-k)!} p^{k+1} (1-p)^{n-k} && \text{Multip. por $1=\frac{(n+1)p}{(n+1)p}$}\\
        &= \frac{1}{(n+1)p}\sum_{k=0}^{n} \frac{(n+1)!}{(k+1)!((n+1)-(k+1))!} p^{k+1} (1-p)^{(n+1)-(k+1)} && \text{Reescribiendo $n-k$}\\
        &= \frac{1}{(n+1)p}\sum_{k=1}^{n+1} \frac{(n+1)!}{k!((n-1)-k)!} p^{k} (1-p)^{(n+1)k} && \text{Cambiando límites de la suma}\\
        &= \frac{1}{(n+1)p} \underbrace{\sum_{k=1}^{n+1} \frac{(n+1)!}{k!((n-1)-k)!} p^{k} (1-p)^{(n+1)k}}_{\text{Consideramos $Bin(n+1,k)$ teniendo así $1 - (1-p)^n$}}
            && \text{Observación previa}\\
        &= \frac{1}{(n+1)p} \cdot (1 - (1-p)^{n+1}) && \text{Sustituyendo}\\
        &= \frac{1 - (1-p)^{n+1}}{(n+1)p}
\end{align*}
\end{proof}

\begin{figure}[H]
    \centering
    \includegraphics[scale=0.25]{triste1.jpg}
    \caption{Siempre que veo una nueva tarea de proba \faFrownO}
\end{figure}

%%%%%%%%%%%%%%% (6)
\item Supón que se realizan $n$ volados donde la probabilidad de que caiga 'Sol' es igual a $p$.
Prueba que la probabilidad de que ocurran un número par de 'Soles' al realizar la serie de lanzamientos es igual a
$\frac{1}{2}[1+(q-p)^n]$ donde $q=1-p$. Primero demuestra y luego utiliza la siguiente identidad
\[
    \sum_{i=0}^{\floor*{\frac{n}{2}}} \binom{n}{2i} p^{2i} q^{n-2i} = 
        \frac{1}{2}[(p+q)^n + (q-p)^n]
\]
Donde $\floor*{\frac{n}{2}}$ es el mayor entero menor igual que $\frac{n}{2}$

\begin{proof}
    Consideremos las siguientes dos igualdades:
    \[
        (p+q)^n = \sum_{k=0}^{n} \binom{n}{k} p^k (1-p)^{n-k}\tag{1}
        \]
    \[
        (p-q)^n = \sum_{k=0}^{n} \binom{n}{k} (-p)^k (1-p)^{n-k}   \tag{2}
    \]
    Si sumamos (1) y (2), entonces todos los términos $k$ son de la forma $k=2i$, teniendo así,
    \[
        (p+q)^n + (p-q)^n = 2 \sum_{k \text{ par}}^{n} \binom{n}{k} (-p)^k (1-p)^{n-k}
    \]
    Reescribiéndola como,
    \[
        \sum_{i=0}^{\floor*{\frac{n}{2}}} \binom{n}{2i} p^{2i} q^{n-2i} = 
            \frac{1}{2}[(p+q)^n + (q-p)^n]
    \]
    Entonces
    \begin{align*}
        P(X = \text{ número par de Soles})
            &= \frac{1}{2} [(p+q)^n + (q-p)^n]\\
            &= \frac{1}{2} [(1 + (q-p)^n]
    \end{align*}


\end{proof}

%%%%%%%%%%%%%%% (7)
\item Sea $X$ una variable aleatoria Poisson con parámetro $\lambda$.
\begin{itemize}
    \item Prueba que: $P(X \text{ es 'Par'}) = \frac{1}{2}[1+e^{-2\lambda}]$
    
    Utiliza el ejercicio anterior y la relación que existe entre una Variable Aleatoria Poisson y la Binomial.

    \begin{proof}
        Recordando que una variable aleatoria $X$ que tomar valores $0,1,2\ldots$ se dice que se distribuye
        \emph{Poisson} con parámetro $\lambda$, si para alguna $\lambda > 0,$
        \[
            p(i) = P(X=i) = e^{-\lambda}\frac{\lambda^{i}}{i!} \qquad i = 0,1,2,\ldots
        \]
        Sea $p\in[0,1]$ y $\lambda=np$ (tomando los parámetros $n$ y $p$ de la Binomial) y tomando la ecuación del 
        inciso anterior, tenemos que
        \[
            P(X \text{ es 'Par'}) = \frac{1}{2}(1+(1-2p)^n) = \frac{1}{2}(1+(1-2\frac{\lambda}{n})^n)
        \]
        Y si aplicamos el límite cuando $n \to \infty$, entonces,
        \[
            \lim_{n \to \infty}P(X \text{ es 'Par'}) = \lim_{n \to \infty} \frac{1}{2}(1+(1-2\frac{\lambda}{n})^n) =
                \frac{1}{2}(1+e^{2\lambda})
        \]
    \end{proof}
    
    \item Prueba el resultado anterior directamente haciendo uso de la expansión en Serie de Taylor de $e^{-\lambda}+e^{\lambda}$.
    
    \begin{proof}
        Usando la expansión de Taylor de una función exponencial se tiene:
        \[
            P(X \text{ es 'Par'}) = P(X = 2k) = \sum_{k=0}^{\infty}\frac{\lambda^{2k}}{(2k)!}e^{-\lambda} =
                e^{-\lambda} \cdot \frac{e^{\lambda} + e^{-\lambda}}{2} = \frac{1 + e^{-2\lambda}}{2}
        \]
        La serie de Taylor de $e^\lambda$ es:
        \[
            e^\lambda = \sum_{k=0}^{\infty} \frac{\lambda^k}{k!} \tag{1}
        \]
        La serie de Taylor de $e^{-\lambda}$ es:
        \[
            e^{-\lambda} = \sum_{k=0}^{\infty} \frac{(-\lambda)^k}{k!} \tag{2}
        \]
        Sumando (1) y (2), se tiene,
        \[
            e^\lambda + e^{-\lambda} =
            \sum_{k=0}^{\infty} \frac{\lambda^k}{k!} + \sum_{k=0}^{\infty} \frac{(-\lambda)^k}{k!} =
            2 \sum_{k=0}^{\infty} \frac{\lambda^{2k}}{(2k)!}
        \]
        Por lo tanto,
        \begin{align*}
            P(X \text{ es 'Par'})
                &= e^{-\lambda}(\sum_{k=0}^{\infty}\frac{\lambda^{2k}}{(2k)!})\\
                &= e^{-\lambda}(\frac{\sum_{k=0}^{\infty}\frac{\lambda^{2k}}{(2k)!}}{2})\\
                &= e^{-\lambda}(\frac{e^\lambda + e^{-\lambda}}{2})\\
                &= \frac{1}{2} (1 + e^{-2\lambda})
        \end{align*}
    \end{proof}
    
\end{itemize}

%%%%%%%%%%%%%%% (8)
\item \faFrownO

%%%%%%%%%%%%%%% (9)
\item Se sabe que en una caseta de la Autopista 'México-Queretaro' los vehículos pasan a razón de 16 carros por minuto.
¿Cual es la probabilidad de que 1000 vehículos atraviesen la caseta en la siguiente hora?

\textit{Hint:} ¿Cuál es la razón por hora?

\sol Dice que pasa 16 autos por minuto pero nos interesa tener cuantos pasan por hora, entonces lo
multiplicqmos por 60, teniendo así que la razón es de 960 autos por hora. Y justamente esa razón de
cambio es nuestro parámetro $\lambda=960$.

Sea $X$ la v.a. que denota el número de autos que atraviesan la caseta por hora.\\
Teniendo así $X \sim Poi(960)$.

Entonces lo que buscamos es $P(X>=1000)$, como no es \emph{fácil} de calcular por medios convencionales,
usaré una calculadora\footnote{\url{https://stattrek.com/online-calculator/poisson.aspx}}.

Por lo tanto $P(X\geq1000)=0.10175$

%%%%%%%%%%%%%%% (10)
\item \faFrownO

%%%%%%%%%%%%%%% (11)
\item \faFrownO

%%%%%%%%%%%%%%% (12)
\item \faFrownO

%%%%%%%%%%%%%%% (13)
\item \faFrownO

%%%%%%%%%%%%%%% (14)
\item \faFrownO

%%%%%%%%%%%%%%% (15)
\item \faFrownO

\begin{figure}[H]
    \centering
    \includegraphics[scale=0.22]{ponido.jpg}
    \caption{Yo había \textit{ponido} las demás respuestas aquí \faFrownO}
\end{figure}

%%%%%%%%%%%%%%% (16)
\item Una escuela compra un lote de 100 focos de los cuales hay 6 defectuosos y los 94 restantes del lote funcionan
correctamente, Si el personal de intendencia elige un foco al azar de este lote para reparar una lampara del plantel y
para ello elige 10 focos de la caja y va probando uno a uno cuales de ellos sirven. Sea $X$ la variable aleatoria
que denota: El número de focos defectuosos encontrados en la muestra. Obten

\sol Recordando la distribución Hipergeométrica, $X \sim Hip(N,n,m)$, entonces,
\[
    P(X=i) = \frac{\binom{m}{i}\cdot\binom{N-m}{n-i}}{\binom{N}{n}}
\]

\begin{itemize}
    \item $P(X = 0)$
    
    \sol
    \[
        P(X=0) = \frac{\binom{6}{0}\binom{94}{100}}{\binom{100}{10}} \approx 0.522
    \]

    \item $P(X > 2)$
    
    \sol
    \begin{align*}
        P(X=2)
            &= 1 - P(X \leq 2) = 1 - [P(X=0) + P(X=1) + P(X=2)]\\
            &= 1 - [0.522 - \frac{\binom{6}{1}\binom{94}{9}}{\binom{100}{10}} - \frac{\binom{6}{2}\binom{94}{8}}{\binom{100}{10}}]\\
            &= 0.0125
    \end{align*}

\end{itemize}
    

\end{enumerate}


%%%%%%%%%%%%%%%%%%%%%%%%%%%%%%%%%%%%%%%%%%%%%%%%%%%%%%%%%%%%%%%%%%%%%%%%%%%%%%%%%%%%%%%%%


%%%%%%%%%%%%%%%%%%%%%%%%%%%%%%%%%%%%%%%%%%%%%%%%%%%%%%%%%%%%%%%%%%%%%%%%%%%%%%%%%%%%%%%%%

%%%%%%%%%%%%%%%%%%%%%%%%%%%%%%%%%%%%%%%%%%%%%%%%%%%%%%%%%%%%%%%%%%%%%%%%%%%%%%%%%%%%%%%%%

\end{document}