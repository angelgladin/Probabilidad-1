%%%%%
    %
    % Copyright (C) 2019 Ángel Iván Gladín García
    %
    % This program is free software: you can redistribute it and/or modify
    % it under the terms of the GNU General Public License as published by
    % the Free Software Foundation, either version 3 of the License, or
    % (at your option) any later version.
    %
    % This program is distributed in the hope that it will be useful,
    % but WITHOUT ANY WARRANTY; without even the implied warranty of
    % MERCHANTABILITY or FITNESS FOR A PARTICULAR PURPOSE.  See the
    % GNU General Public License for more details.
    %
    % You should have received a copy of the GNU General Public License
    % along with this program.  If not, see <http://www.gnu.org/licenses/>.
%%%%%

%%%%%%%%%%%%%%%%%%%%%%%%%%%%%%%%%%%%%%%%%%%%%%%%%%%%%%%%%%%%%%%%%%%%%%%%%%%%%%%%%%%%%%%%%
\documentclass[11pt,letterpaper]{report}
\usepackage[margin=.7in]{geometry}
\usepackage[utf8]{inputenc}
\usepackage[spanish]{babel}
\decimalpoint

\usepackage{listings}
\usepackage{color}
\usepackage{graphicx}
\usepackage{enumerate}
\usepackage{enumitem}
\usepackage{float}

\usepackage{longtable}
\usepackage{hyperref}
\usepackage{commath}

\usepackage{bbm}
\usepackage{cancel}
\usepackage{dsfont}
\usepackage{mathrsfs}
\usepackage{amsmath,amsthm,amssymb}
\usepackage{mathtools}
\usepackage{longtable}

\usepackage{tikz}
\usetikzlibrary{trees}
\usepackage{verbatim}

% Set the overall layout of the tree
\tikzstyle{level 1}=[level distance=3.5cm, sibling distance=3.5cm]
\tikzstyle{level 2}=[level distance=3.5cm, sibling distance=2cm]

% Define styles for bags and leafs
\tikzstyle{bag} = [text width=4em, text centered]
\tikzstyle{end} = [circle, minimum width=3pt,fill, inner sep=0pt]
%%%%%%%%%%%%%%%%%%%%%%%%%%%%%%%%%%%%%%%%%%%%%%%%%%%%%%%%%%%%%%%%%%%%%%%%%%%%%%%%%%%%%%%%%%%%%%%%
\DeclarePairedDelimiter\floor{\lfloor}{\rfloor}
\usepackage{import}

\usepackage[utf8]{inputenc}

\usepackage{listings}
\usepackage{color}

\definecolor{codegreen}{rgb}{0,0.6,0}
\definecolor{codegray}{rgb}{0.5,0.5,0.5}
\definecolor{codepurple}{rgb}{0.58,0,0.82}
\definecolor{backcolour}{rgb}{0.95,0.95,0.92}

\lstdefinestyle{mystyle}{
    backgroundcolor=\color{backcolour},   
    commentstyle=\color{codegreen},
    keywordstyle=\color{magenta},
    numberstyle=\tiny\color{codegray},
    stringstyle=\color{codepurple},
    basicstyle=\footnotesize,
    breakatwhitespace=false,         
    breaklines=true,                 
    captionpos=b,                    
    keepspaces=true,                 
    numbers=left,                    
    numbersep=5pt,                  
    showspaces=false,                
    showstringspaces=false,
    showtabs=false,                  
    tabsize=2
}

\lstset{style=mystyle}
%%%%%%%%%%%%%%%%%%%%%%%%%%%%%%%%%%%%%%%%%%%%%%%%%%%%%%%%%%%%%%%%%%%%%%%%%%%%%%%%%%%%%%%%%


%%%%%%%%%%%%%%%%%%%%%%%%%%%%%%%%%%%%%%%%%%%%%%%%%%%%%%%%%%%%%%%%%%%%%%%%%%%%%%%%%%%%%%%%%
\newcommand{\Z}{\mathbb{Z}}
\newcommand{\N}{\mathbb{N}}
\newcommand{\Q}{\mathbb{Q}}
\newcommand{\R}{\mathbb{R}}
\newcommand{\Pro}{\mathds{P}}
\newcommand{\Oh}{\mathcal{O}} %% Notacion "O"
\newcommand{\lra}{\longrightarrow}
\newcommand{\ra}{\rightarrow}
\newcommand{\ord}{\text{ord}}
\newcommand{\sol}{\textbf{\underline{Solución}: }} %% Solucion
\newcommand{\af}{\textbf{\underline{Afirmación}: }}
\newcommand{\cej}{\textbf{\underline{Contraejemplo}: }}

%%%%%%%%%%%%%%%%%%%%%%%%%%%%%%%%%%%%%%%%%%%%%%%%%%%%%%%%%%%%%%%%%%%%%%%%%%%%%%%%%%%%%%%%%

\begin{document}

%%%%%%%%%%%%%%%%%%%%%%%%%%%%%%%%%%%%%%%%%%%%%%%%%%%%%%%%%%%%%%%%%%%%%%%%%%%%%%%%%%%%%%%%%
\title{
        Universidad Nacional Autónoma de México\\
        Facultad de Ciencias\\
        Probabilidad I\\
    \vspace{1cm}
    \large
        \textbf{Tarea 5}\\
        \textbf{Variables Aleatorias Discretas II}
}
\author{
    Ángel Iván Gladín García\\
    No. cuenta: 313112470\\
    \texttt{angelgladin@ciencias.unam.mx}
}
\date{9 de noviembre 2019}
\maketitle
%%%%%%%%%%%%%%%%%%%%%%%%%%%%%%%%%%%%%%%%%%%%%%%%%%%%%%%%%%%%%%%%%%%%%%%%%%%%%%%%%%%%%%%%%

%%%%%%%%%%%%%%%%%%%%%%%%%%%%%%%%%%%%%%%%%%%%%%%%%%%%%%%%%%%%%%%%%%%%%%%%%%%%%%%%%%%%%%%%%
\newtheorem{theorem}{Teorema}
\newtheorem{example}{Ejemplo}
\newtheorem{corollary}{Corolario}
\newtheorem{lemma}{Lemma}
\newtheorem{definition}{Definicion}
\newtheorem{prop}{Proposicion}
%%%%%%%%%%%%%%%%%%%%%%%%%%%%%%%%%%%%%%%%%%%%%%%%%%%%%%%%%%%%%%%%%%%%%%%%%%%%%%%%%%%%%%%%%

%%%%%%%%%%%%%%%%%%%%%%%%%%%%%%%%%%%%%%%%%%%%%%%%%%%%%%%%%%%%%%%%%%%%%%%%%%%%%%%%%%%%%%%%%
\begin{enumerate}

%%%%%%%%%%%%%%% (1)
\item Si cada persona en una comunidad de 1000 habitantes tiene un 1\% de chances de resultar
infectado de un peligroso virus. Suponiendo que una persona resulta infectada independientemente de
lo que ocurra con otra persona de la comunidad. Calcula la probabilidad de que:
\begin{itemize}
    \item Exactamente haya 10 personas infectadas
    
    \sol
    
    \item A lo mas haya 16 personas infectadas
    
    \sol

    \item Haya entre 12 y 14 personas infectadas
    
    \sol

    \item Alguien resulte infectado
    
    \sol

\end{itemize}

%%%%%%%%%%%%%%% (2)
\item En 1693 Samuel Pepys le escribió una carta a Isaac Newton en donde le planteo el siguiente
problema: ¿Cuál de los siguientes tres planteamientos es mas probable que ocurra?
\begin{itemize}
    
    \item Se lanzan 6 dados y aparece al menos un 6's
    
    \sol

    \item Se lanzan 12 dados y aparecen al menos dos 6’s
    
    \sol
    
    \item Se lanzan 18 dados y aparecen al menos tres 6’s
    
    \sol
\end{itemize}
    
Como dato adicional Sir Isaac Newton ademas de ser uno de los mas grandes científicos de la historia,
durante un gran lapso de su vida fue un funcionario publico, manejo digamos el SAT de lo que hoy es
el Reino Unido. Y estaba a favor usar la horca o quemar gente si evadian impuestos, eran buenos tiempos.

%%%%%%%%%%%%%%% (3)
\item Tenemos 2 dados 'A' y 'B', uno de los cuales es standard y el otro tiene 3 caras con 1's y 4
caras con 4's. Se lanza una moneda: Si cae \emph{Águila} lanzamos el dado 'A' 5 veces mientras que si cae
\emph{Sol} lanzamos el dado 'B' 5 veces. Sea $X$ la variable aleatoria que denota: El número de veces que
el '4' aparece

\begin{itemize}
    \item Obtén P(X=3)
    
    \sol

    \item ¿Tendrá $X$ una Distribución Binomial?
    
    \sol
    
\end{itemize}

%%%%%%%%%%%%%%% (4)
\item Un viejo juego de dados llamado \emph{Chuck a Luck} popular en el EE.UU. del S XIX. Consiste en que
un jugador apuesta a algún número entre el 1 y el 6 entonces se lanzan 3 dados. Si el número al que le aposto el jugador
aparece una, dos o tres veces al lanzar los dados el jugador recibe una, dos o tres veces lo que aposto, mas su apuesta
inicial y pierde su apuesta inicial en cualquier otro caso. Sea $X$ la variable aleatoria que denota: La ganancia del
jugador por unidad del principal apostado inicialmente.

\begin{itemize}
    \item Obtén la función de masa de probabilidad asociada a $X$.
    
    \sol

    \item Calcula el Valor esperado de la ganancia del jugador por unidad apostada.
    
    \sol
    
\end{itemize}

%%%%%%%%%%%%%%% (5)
\item Si $X \sim Bin(n, p)$.

Prueba que. $\mathds{E}(\frac{1}{X+1}) = \frac{1-(1-p)^{n-1}}{(n+1)p}$

\begin{proof}
    
\end{proof}

%%%%%%%%%%%%%%% (6)
\item Supón que se realizan $n$ volados donde la probabilidad de que caiga 'Sol' es igual a $p$.
Prueba que la probabilidad de que ocurran un número par de 'Soles' al realizar la serie de lanzamientos es igual a
$\frac{1}{2}[1+(q-p)^n]$ donde $q=1-p$. Primero demuestra y luego utiliza la siguiente identidad
\[
    \sum_{i=0}^{\floor*{\frac{n}{2}}} \binom{n}{2i} p^{2i} q^{n-2i} = 
        \frac{1}{2}[(p+q)^n + (q-p)^n]
\]
Donde $\floor*{\frac{n}{2}}$ es el mayor entero menor igual que $\frac{n}{2}$

\begin{proof}
    
\end{proof}

%%%%%%%%%%%%%%% (7)
\item Sea $X$ una variable aleatoria Poisson con parámetro $\lambda$.
\begin{itemize}
    \item Prueba que: $P(X \text{ es 'Par'}) = \frac{1}{2}[1+e^{-2\lambda}]$
    
    Utiliza el ejercicio anterior y la relación que existe entre una Variable Aleatoria Poisson y la Binomial.

    \begin{proof}
        
    \end{proof}
    
    \item Prueba el resultado anterior directamente haciendo uso de la expansión en Serie de Taylor de $e^{-\Lambda}+e^{\lambda}$.
    
    \begin{proof}
        
    \end{proof}
    
\end{itemize}

%%%%%%%%%%%%%%% (8)
\item Si $X \sim Poisson(\lambda)$

\begin{itemize}    
    \item Para que valores de $\lambda$ el valor de $P(X=i)$, $i \geq 0$ es máximo.
    
    \sol
    
    \item Prueba que: $\mathds{E}(X^{n}) = \lambda \mathds{E}[(X+1)^{n-1}]$ y utiliza este resultado para calcular el
    valor de $\mathds{E}(X^3)$

    \begin{proof}
        
    \end{proof}

\end{itemize}

%%%%%%%%%%%%%%% (9)
\item Se sabe que en una caseta de la Autopista 'México-Queretaro' los vehículos pasan a razón de 16 carros por minuto.
¿Cual es la probabilidad de que 1000 vehículos atraviesen la caseta en la siguiente hora?

\textit{Hint:} ¿Cuál es la razón por hora?

\sol

%%%%%%%%%%%%%%% (10)
\item  Jennifer en una visita al Otorrinorrinolaringologo, el medico le recomienda tomar unos suplementos para evitar
que ella enferme de las vías respiratorias durante el próximo año. Le recomienda seguir una terapia experimental en
donde ella consumirá alguno de 2 suplementos disponibles durante un tiempo.

Si toma el Suplemento A , El número de veces que Jenny enfermara se espera que sea 1 vez al año posterior a terminar el
tratamiento mientras que si toma el suplemento B, ella se espera que enferme en 4 ocasiones. El medico tiene igual
numero de cajas del Suplemento A y B. Y puede elegir cualquiera de ellos con igual probabilidad para darselos a su
paciente. Si ella se enfermo 3 veces durante el año siguiente a seguir alguno de los tratamientos. ¿Cuál es la
probabilidad de que ella haya tomado el suplemento B?

\sol

%%%%%%%%%%%%%%% (11)
\item Schneider una conocida manufacturera produce componentes eléctricos de los cuales aproximadamente el 1\% de ellos
son defectuosos. Si el departamento de control de calidad elige una gran cantidad de ellos de manera independiente uno
de otros. ¿Cuál es la probabilidad de que:

\begin{itemize}
    \item Hayan revisado 110 componentes antes de que hayan encontrado uno defectuoso?
    
    \sol

    \item Hayan revisado a lo mas 110 componentes antes de que hayan encontrado uno defectuoso?
    
    \sol

    \item Obtén el Numero esperado de componentes que tengan que revisar para que encuentren un componente defectuoso?
    
    \sol

\end{itemize}

%%%%%%%%%%%%%%% (12)
\item Una Caja tiene $r$ pelotas rojas y $b$ pelotas azules. Si elegimos una pelota a la vez y sin reemplazo. Sea $X$
la variable aleatoria que denota el número de extracciones que tenemos que hacer de la urna hasta que la primer bola
roja es seleccionada.

\begin{itemize}
    \item Explica por que $X$ no tiene una distribución geométrica
    
    \sol

    \item Demuestra que la $p_X(i)$ esta dada por:
    $p_X(i)= \frac{binom{r+b-i}{r-1}}{\binom{r+b}{r}}$ si $i=1,2,\ldots,b+1$

    \begin{proof}
        
    \end{proof}
\end{itemize}

%%%%%%%%%%%%%%% (13)
\item Si la función generadora de momentos de una Variable Aleatoria Discreta esta definida como:
\[
    M(t) = \mathds{E}(e^{tX}) = \sum_{x} e^{tX} p(x)
\]
Si $X \sim Geometrica(p)$ Prueba que:
\begin{itemize}
    \item $M_X(t) = \frac{pe^t}{1-(1-p)e^t}$
    
    \begin{proof}
        
    \end{proof}

    \item Utiliza lo anterior para demostrar que: $\mathds{E}(X) = \frac{1}{p}$ y $Var(X) = \frac{1 - p}{p^2}$
    
    \begin{proof}
        
    \end{proof}
\end{itemize}

%%%%%%%%%%%%%%% (14)
\item Si se lanza una moneda honesta hasta que sale Sol por 10ma. Vez. Sea $X$ la variable aleatoria que denota: El
número de lanzamientos necesarios hasta que salen 10 "Soles". Obtén la función de masa de probabilidad asociada a X y
el Número de lanzamientos que se espera tengamos que realizar para obtener dicho resultado.

\sol

%%%%%%%%%%%%%%% (15)
\item Supón que una muestra de tamaño $n$ es seleccionada al azar y sin reemplazo de una urna que contiene $N$ pelotas de
las cuales $m$ son blancas y $M-N$ son negras. Sea $X$ la variable aleatoria que denota: El número de pelotas
blancas seleccionadas.

\begin{itemize}
    \item Obtén la función de masa de probabilidad de la Variable Aleatoria $X$
    
    \sol

    \item Calcula $\mathds{E}(X)$ y $Var(X)$
    
    \sol

\end{itemize}

%%%%%%%%%%%%%%% (16)
\item Una escuela compra un lote de 100 focos de los cuales hay 6 defectuosos y los 94 restantes del lote funcionan
correctamente, Si el personal de intendencia elige un foco al azar de este lote para reparar una lampara del plantel y
para ello elige 10 focos de la caja y va probando uno a uno cuales de ellos sirven. Sea $X$ la variable aleatoria
que denota: El número de focos defectuosos encontrados en la muestra. Obten

\begin{itemize}
    \item $P(X = 0)$
    
    \sol

    \item $P(X > 2)$
    
    \sol

\end{itemize}
    

\end{enumerate}


%%%%%%%%%%%%%%%%%%%%%%%%%%%%%%%%%%%%%%%%%%%%%%%%%%%%%%%%%%%%%%%%%%%%%%%%%%%%%%%%%%%%%%%%%


%%%%%%%%%%%%%%%%%%%%%%%%%%%%%%%%%%%%%%%%%%%%%%%%%%%%%%%%%%%%%%%%%%%%%%%%%%%%%%%%%%%%%%%%%

%%%%%%%%%%%%%%%%%%%%%%%%%%%%%%%%%%%%%%%%%%%%%%%%%%%%%%%%%%%%%%%%%%%%%%%%%%%%%%%%%%%%%%%%%

\end{document}