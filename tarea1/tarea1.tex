%%%
 %
 % Copyright (C) 2019 Ángel Iván Gladín García
 %
 % This program is free software: you can redistribute it and/or modify
 % it under the terms of the GNU General Public License as published by
 % the Free Software Foundation, either version 3 of the License, or
 % (at your option) any later version.
 %
 % This program is distributed in the hope that it will be useful,
 % but WITHOUT ANY WARRANTY; without even the implied warranty of
 % MERCHANTABILITY or FITNESS FOR A PARTICULAR PURPOSE.  See the
 % GNU General Public License for more details.
 %
 % You should have received a copy of the GNU General Public License
 % along with this program.  If not, see <http://www.gnu.org/licenses/>.
%%%

%%%%%%%%%%%%%%%%%%%%%%%%%%%%%%%%%%%%%%%%%%%%%%%%%%%%%%%%%%%%%%%%%%%%%%%%%%%%%%%%%%%%%%%%%
\documentclass[11pt,letterpaper]{report}
\usepackage[margin=1in]{geometry}
\usepackage[utf8]{inputenc}
\usepackage[spanish]{babel}
 
\usepackage{listings}
\usepackage{color}
\usepackage{graphicx}
\usepackage{enumerate}
\usepackage{enumitem}
\usepackage{float}

\usepackage{longtable}
\usepackage{hyperref}
\usepackage{commath}

\usepackage{bbm}
\usepackage{dsfont}
\usepackage{mathrsfs}
\usepackage{amsmath,amsthm,amssymb}
\usepackage{mathtools}
\usepackage{longtable}
%%%%%%%%%%%%%%%%%%%%%%%%%%%%%%%%%%%%%%%%%%%%%%%%%%%%%%%%%%%%%%%%%%%%%%%%%%%%%%%%%%%%%%%%%%%%%%%%5

\usepackage{import}

\usepackage[utf8]{inputenc}
 
\usepackage{listings}
\usepackage{color}
 
\definecolor{codegreen}{rgb}{0,0.6,0}
\definecolor{codegray}{rgb}{0.5,0.5,0.5}
\definecolor{codepurple}{rgb}{0.58,0,0.82}
\definecolor{backcolour}{rgb}{0.95,0.95,0.92}
 
\lstdefinestyle{mystyle}{
    backgroundcolor=\color{backcolour},   
    commentstyle=\color{codegreen},
    keywordstyle=\color{magenta},
    numberstyle=\tiny\color{codegray},
    stringstyle=\color{codepurple},
    basicstyle=\footnotesize,
    breakatwhitespace=false,         
    breaklines=true,                 
    captionpos=b,                    
    keepspaces=true,                 
    numbers=left,                    
    numbersep=5pt,                  
    showspaces=false,                
    showstringspaces=false,
    showtabs=false,                  
    tabsize=2
}
 
\lstset{style=mystyle}
%%%%%%%%%%%%%%%%%%%%%%%%%%%%%%%%%%%%%%%%%%%%%%%%%%%%%%%%%%%%%%%%%%%%%%%%%%%%%%%%%%%%%%%%%


%%%%%%%%%%%%%%%%%%%%%%%%%%%%%%%%%%%%%%%%%%%%%%%%%%%%%%%%%%%%%%%%%%%%%%%%%%%%%%%%%%%%%%%%%
\newcommand{\Z}{\mathbb{Z}}
\newcommand{\N}{\mathbb{N}}
\newcommand{\Q}{\mathbb{Q}}
\newcommand{\R}{\mathbb{R}}
\newcommand{\Oh}{\mathcal{O}} %% Notacion "O"
\newcommand{\lra}{\longrightarrow}
\newcommand{\ra}{\rightarrow}
\newcommand{\ord}{\text{ord}}
\newcommand{\sol}{\textbf{\underline{Solución}: }} %% Solucion

%%%%%%%%%%%%%%%%%%%%%%%%%%%%%%%%%%%%%%%%%%%%%%%%%%%%%%%%%%%%%%%%%%%%%%%%%%%%%%%%%%%%%%%%%

\begin{document}

%%%%%%%%%%%%%%%%%%%%%%%%%%%%%%%%%%%%%%%%%%%%%%%%%%%%%%%%%%%%%%%%%%%%%%%%%%%%%%%%%%%%%%%%%
\title{
        Universidad Nacional Autónoma de México\\
        Facultad de Ciencias\\
        Probabilidad I\\
    \vspace{1cm}
    \large
        \textbf{Tarea 1}\\
        \textbf{Técnicas de Conteo}
}
\author{
    Ángel Iván Gladín García\\
    No. cuenta: 313112470\\
    \texttt{angelgladin@ciencias.unam.mx}
}
\date{29 de Agosto 2019}
\maketitle
%%%%%%%%%%%%%%%%%%%%%%%%%%%%%%%%%%%%%%%%%%%%%%%%%%%%%%%%%%%%%%%%%%%%%%%%%%%%%%%%%%%%%%%%%

%%%%%%%%%%%%%%%%%%%%%%%%%%%%%%%%%%%%%%%%%%%%%%%%%%%%%%%%%%%%%%%%%%%%%%%%%%%%%%%%%%%%%%%%%
\newtheorem{theorem}{Teorema}
\newtheorem{example}{Ejemplo}
\newtheorem{corollary}{Corolario}
\newtheorem{lemma}{Lemma}
\newtheorem{definition}{Definicion}
\newtheorem{prop}{Proposicion}
%%%%%%%%%%%%%%%%%%%%%%%%%%%%%%%%%%%%%%%%%%%%%%%%%%%%%%%%%%%%%%%%%%%%%%%%%%%%%%%%%%%%%%%%%


%%%%%%%%%%%%%%%%%%%%%%%%%%%%%%%%%%%%%%%%%%%%%%%%%%%%%%%%%%%%%%%%%%%%%%%%%%%%%%%%%%%%%%%%%
\begin{enumerate}

\item Demuestra la version generalizada del Principio Básico del Conteo.\\
\textit{Hint:} Utiliza las Propiedades del Producto Cartesiano de dos conjuntos A, B.\\\\
\textit{El Principio Básico del Conteo Generalizado: } Si hay $r$ experimentos que se harán,
y el primero de ellos tiene $n_1$ posibles resultados, y por cada uno de estos $n_1$ resultados
hay $n_2$ posibles resultados del segundo experimento y así sucesivamente, entonces hay un
total de $n_1 \cdot n_2 \cdot \ldots \cdot n_r$ posibles resultados de $r$ experimentos.
\begin{proof}
    \underline{Por inducción}\\
    \textbf{Caso base:} Para $r=1$, como solo un experimento se hace y da $n_1$ resultados y
    evidentemente hay $n_1$ resultados de un experimento.\\
    \textbf{Hipótesis de inducción:} Para $r=n$ supongamos que la generalización del
    principio básica del conteo es verdadera para un número $r$ específica de experimentos.\\
    \textbf{Paso inductivo: } Para $r=n+1$. Si se hacen $r$ experimentos por hipótesis de inducción
    hay $n_1 \cdot n_2 \cdot \ldots \cdot n_r$ resultados. Por el principio básico del conteo entonces
    para $r=n+1$ entonces serían $n_1 \cdot n_2 \cdot \ldots \cdot n_r \cdot (n_{r+1})$ posibles
    resultados.\\
    Por el principio de inducción es verdadera para cualquier $r \in \N$.
\end{proof}

\item Una compañía de telecomunicaciones desea saber cual es la cantidad de números telefónicos de 7
cifras dispone para telefonía fija si:
\begin{enumerate}[label=\alph*)]
    \item La primer cifra no puede ser 0 ó 1.\\
    \sol Por el principio de las casillas, debemos de tener 7 casillas. En la primera de ellas,
    sea $E_1 = \{ 2,3,4,5,6,7,8,9\}$ y $E_i = \{n \mid 0 \leq n \leq 9\}$ con
    $2 \leq i \leq  9$. Dicho esto hay
    $\#E_1 \cdot \#E_2 \cdot \ldots \cdot \#E_7 = 8 \cdot 10^{6} = 8,000,000$.

    \item Resuelve el ejercicio anterior de nuevo asumiendo que un número telefónico no puede
    iniciar con 911 ya que esta marcación esta reservada para emergencias\\
    \sol Por el princio de las casillas, si fijamos las tres primeras casillas con 9, 9 y 1,
    entonces quedan 4 casillas restantes con 10 opciones cada una, quedadando así que hay
    $10^4$ opciones de teléfonos que inician con 911, llamemos al experimento anterior
    $E$ con $\#E=10^4$. Por el inciso anterior tenemos a $F$ con $\#F = 8,000,000$, lo único
    que se debe de hacer es restarlos, teniendo así $\#F - \#E = 7,990,000$.

\end{enumerate}

\item Un niño tiene 12 bloques de los cuales 6 son de color negro, 4 son rojos, un blanco y otro azul.
Si el niño acomoda los bloques sobre una línea. ¿Cuántos acomodos distintos de los bloques puede hacer?\\
\sol Lo primero que hay que hacer es calcular todas la posibles permutaciones de los bloques,
por el momento sin importar los colores, que son $12!$. Ahora dividirlo entre el número de
bloques que hay de cada color, quedando así:
$$\frac{12!}{6! \cdot 4! \cdot 1! \cdot 1!} = 27720$$

\item ¿De cuántas maneras podemos acomodar en una repisa, 3 libros de literatura, 2 de matemáticas y 
otro libro de química si.
\begin{enumerate}[label=\alph*)]
    \item los libros pueden ser acomodados en cualquier orden?\\
    \sol Si sí nos importa el contenido de los libros hay $6! = 720$ de acomodarlas (sin
    restricciones), pero si no, ósea solo saber el tipo de libro en específico hay
    $\frac{6!}{3! \cdot 2! \cdot 1!} = 60$.

    \item los libros de matemáticas tienen que esta juntos y los de literatura también?\\
    \sol Como los libros de matemáticas tiene que estar juntos, se pueden ver como un lote de
    libros de matemáticas, y cualquier oreden de ese lote son las permutaciones que es $2!$.
    De manera análoga con los libros de literatura, habíendo $3!$.
    Si vemos los libros de literatura, matemáticas y química como un lote cada uno, 
    se sigue que hay $3!$ formas de acomodar los lotes.\\
    Con el análisis previo hay $3! \cdot 2! \cdot 3! \cdot 1! = 72$ maneras.

    \item los de literatura siempre tienen que estar juntos, pero los de los otros temas pueden estar
    acomodados en cualquier orden?\\
    \sol Como los libros de literatura deben de estar juntos, los podemos considerar como un lote,
    teniendo así que las formas de ordenar los libros de ese lote son $4!$. Como quedan los
    otros 3 libros restantes, la formas de ordenarlos son $3!$.\\
    Quedando así $4! \cdot 3! = 144$ diferentes maneras.
\end{enumerate}

\item Dos amigos $A$ y $B$ pasan las tardes jugando FIFA 19 en su PlayStation 4. Ellos jugaran 7 partidos,
cada juego tiene 3 posibles resultados: Victoria para $A$ (lo cual es una derrota para $B$), 
Empate y Victoria para $B$ (lo cual es una derrota para $A$). De modo que una victoria para $A$
equivale a 1 punto, un empate equivale a $\frac{1}{2}$ punto mientras que una derrota equivale
a 0 puntos. ¿De cuántas maneras
\begin{enumerate}[label=\alph*)]
    \item al jugador $A$ le es posible terminar la serie de 7 partidos, con 3 victorias, 2 empates y 2 derrotas?\\
    \sol $\frac{7!}{3! \cdot 2! \cdot 2!} = 210$ maneras.

    \item Al jugador $A$ le es posible terminar la serie de 7 partidos con 4 puntos y al jugador B con 3 puntos?\\
    \sol De $\frac{7!}{4! \cdot 3!} = 35$ maneras distintas.
\end{enumerate}

\item Un estudiante venderá 2 libros de su coleccion de libros la cual esta integrada por 6 libros de
matemáticas, 7 de física y 4 de economía. ¿De cuántas maneras lo puede hacer si:
\begin{enumerate}[label=\alph*)]
    \item ambos libros tienen que ser del mismo tema?\\
    \sol Si toma dos libros de matemáticas, 2 de física y 4 de economía, en total son:
    $$\binom{6}{2} + \binom{7}{2} + \binom{4}{2} = 42$$
    \item los libros son de temas distintos?\\
    \sol Tomando cada pareja posible de dos temas de libros posibles se tienen $\binom{3}{2} = 3$
    que son: \{matemáticas, fisica\}, \{matemáticas, economía\} y \{física, economía\}.
    Usando el principio de las casillas (dos casillas por cada experimento) se sigue que son:
    $$6 \cdot 7 + 6 \cdot 4 + 7 \cdot 4 = 94$$
\end{enumerate}

\item Un comité de 3 ingenieros eléctricos y 3 ingenieros mecánicos será elegido de la plantilla
total de ingenieros que tiene una fábrica la cual consta de 7 ingenieros eléctricos y 5 ingenieros
mecánicos. Obten el número de formas de elegir este comité de ingenieros si:
\begin{enumerate}[label=\alph*)]
    \item cualquier ingeniero eléctrico y mecánico puede ser seleccionado.\\
    \sol Para obtener de cuantas formas podemos obtener 7 ingerieros eléctricos de 3 formas distintas
    se tiene que es $\binom{7}{3}$ y para los mecánicos son $\binom{5}{3}$, por el por el principio
    de las casillas son:
    $$\binom{7}{3} \cdot \binom{5}{3} = 350$$

    \item un ingeniero eléctrico en particular debe de ser miembro del comité.\\
    \sol Tomando a un ingeniero eléctrico en particular que tomamos uno y al realizar las
    posibles combinaciones quitamos a un miembro y solo tomamos 2, que es $\binom{6}{2}$.
    Depués obtenemos cuantas combinaciones se tienen de los ingenieros mecánicos que son
    $\binom{5}{3}$. Por último, como al princio dejamos a un ingeniero eléctrico fijo,
    las posibles combinaciones de éste son $1!$. Ergo, las posibles posibles formas son:
    $$\binom{6}{2} \cdot \binom{5}{3} \cdot 1! = 150$$

    \item dos ingenieros mecanicos en particular no pueden ser miembros del comité.\\
    \sol De manera análoga al inciso anterior, fijamos a dos ingenieros eléctrico, hecho esto
    nos queda una sola opción para el otro ing. mecánico. Depues para obtener la combinaciones
    del ingeniero mecánico son $\binom{7}{3}$. Quedadndo así:
    $$\binom{3}{3} \cdot \binom{7}{3} = 35$$
\end{enumerate}


\item Determina el número de vectores $(x_1,x_2,\ldots,x_n)$ tal que $x_i$ es igual a 0 ó 1
y $\sum_{i=1}^{n} x_i \geq k$.\\
\sol Si cada $x_i$, es 0 o 1, la suma de las entradas del vector en $\sum_{i=1}^{n} x_i \geq k$ es
solo el número de entradas que son 1. Se sigue que el el número de vectores donde la suma
de las entradas es $k$ es igual al número de vectores que tienen $k$ entradas igual a 1,
que es $\binom{n}{k}$. El número de vectores donde la suma es al menos $k$ es igual a la suma
del número de vectores donde la suma es exactamente $l$ de $k$ a $n$. Quedadndo así 
$\sum_{l=k}^{n} \binom{n}{l}$.

\item Una colección de arte en subasta consta de 4 Dali's, 5 Van Gogh's y 6 Picasso's. A la subasta
acudieron 5 coleccionistas de arte asi un auditor toma nota del numero de Dali's, Van Gogh's y 
Picasso's adquiridos por cada coleccionista.\\
¿Cuántos registros distintos puede realizar el auditor si todos los cuadros fueron vendidos?
\sol Lo primero que hay que obtener es el número de todas las formas que se pueden vender todas
las obras de cada uno; que es, todas las formas en que que se pudieron vender las obras 4 obras
de Dali, las 5 de Van Gogh y las 6 de Picasso\\
Para esto hay que recordar que para obener todas las formas\footnote{Proposición 6.2 de la página 13
del libro \textit{A First Course in Probability}, Sheldon Ross.} que hay de $n$ para $k$ son:
$$\binom{n+k-1}{k-1}$$
Entonces para obtener todas los posibles de cada obra de Dali, Van Gogh y Picasso se tiene
respectivamente que son: $\binom{4+1-1}{5-1}$, $\binom{5+5-1}{5-1}$ y $\binom{6+5-1}{5-1}$.\\
Ergo, la forma en la que se pueden resgitrar son:
$$\binom{8}{4} \cdot \binom{9}{4} \cdot \binom{10}{4} = 1,852,200$$

\item
\begin{enumerate}[label=\alph*)]
    \item Prueba que $\binom{n}{0} + \binom{n+1}{1} + \cdots + \binom{n+r}{r} = \binom{n+r+1}{r}$\\
    \textit{Hint:} $\binom{n}{r} = \binom{n+1}{r} - \binom{n}{r-1}$
    \begin{proof}
        TODO
    \end{proof}

    \item Obten $\binom{n}{1} + 2\binom{n}{2} + 3\binom{n}{3} + \ldots + n\binom{n}{n}$\\
    \textit{Hint:} Escribe de manera adecuada $i\binom{n}{i}$
    \begin{proof}
        Reescribiendo la expresión se tiene que es
        $$\binom{n}{1} + 2\binom{n}{2} + 3\binom{n}{3} + \ldots + n\binom{n}{n} =
        \sum_{i=1}^n k \binom{n}{i}$$
        Y también hay una identidad que dice lo siguiente:
        $$\sum_{i=1}^n i \binom{n}{i} = n \cdot 2^{n-1}$$
        Se tratará este problema como si fueran comités:\\
        Dadas $n$ personas, se verán de cuantas formas se puede seleccionar un comité
        de un grupo, donde cada comté tiene un presidente. Lo primero que se hará
        es fijar un tamaño $k$ de cada comité, tomar $k$ miembros de $\binom{n}{k}$
        maneras, y luego nombrar un presidente de entre los miembros seleccionados
        con $k$ opciones, dando un total de $\sum_{k=1}^{n} k \cdot \binom{n}{k}$
        maneras de seleccionar un comité.

        O de manera análoga, primero seleccionar al presidente con $n$ posibles
        opciones, y después seleccionar a lo miembros restantes en $2^{n-1}$
        maneras. Dando un total de $n \cdot 2^{n-1}$ formas de tomar un comité.

        Teniendo así la igualdad:
        $$\sum_{k=1}^{n} k \cdot \binom{n}{k} = n \cdot 2^{n-1}$$
    \end{proof}
\end{enumerate}

\item De un grupo de 8 mujeres y 6 hombres que trabajan en una oficina se elegirán un grupo que será
integrado por 3 hombres y 3 mujeres para comisión interna de Protección Civil. ¿Cuántos coomités
es posible formar si:
\begin{enumerate}[label=\alph*)]
    \item dos hombres en específico se niegan a pertenecer al comité juntos?\\
    \sol El comité que no tiene a dos hombres es
    $\binom{8}{3} \cdot \binom{4}{3}$. Para obtener el número de comités que
    incluye uno de los hombres que se niega son
    $\binom{8}{3} \cdot \binom{4}{2} \cdot \binom{2}{1}$.

    Por tanto el número de posibilidades que que dos hombre en especifico se nieguen a
    pertenecer al comité son:
    $$\binom{8}{3} \cdot \binom{4}{3} + \binom{8}{3} \cdot \binom{4}{2} \cdot \binom{2}{1} = 896$$


    \item dos mujeres en específico se niegan a pertenecer al comité juntas?\\
    \sol Si el comité no incluye a dos mujeres que se nieguen hay
    $\binom{6}{3} \cdot \binom{6}{3}$. Para obtener el número de comités que incluyen
    solo a una de las mujeres que se niegan a pertenecer son
    $\binom{6}{2} \cdot \binom{2}{1} \cdot \binom{6}{3}$.

    Por tanto el número de comités a formar son:
    $$\binom{6}{3} \cdot \binom{6}{3} +\binom{6}{2} \cdot \binom{2}{1} \cdot \binom{6}{3} = 1000$$

    \item una mujer y un hombre en específico se niegan a pertenecer al comité juntos?\\
    \sol Hay $\binom{7}{3} \cdot \binom{5}{3}$ posibles comités cuando ninguno de los
    2 quieren. Cuando una mujer en específico se niega a participar hay
    $\binom{7}{2} \cdot \binom{5}{3}$. Cuando un hombre en específico se niega a
    participar hay $\binom{7}{3} \cdot \binom{5}{2}$.

    Por lo tanto el número de comités que se pueden formar son:
    $$\binom{7}{3} \cdot \binom{5}{3} + \binom{7}{2} \cdot \binom{5}{3} + \binom{7}{3} \cdot \binom{5}{2} = 910$$
\end{enumerate}

\item En una reunión del Consejo de Seguridad de la O.N.U. delegados de 10 países donde están incluidos:
Rusia, China, Estados Unidos e Irán serán ubicados en una fila de silla para dar una conferencia de
prensa conjunta. ¿De cuántas maneras los pueden acomodar si los delegados de Rusia e Irán siempre
tienen que estar juntos y el representante de China y EE.UU. no pueden estar juntos?

\sol Hay 10 delegados, los delegados de Irán y Rusia siempre van a estar juntos, considerandolos
como una sola entidad (un solo delegado) hay en total 9 ``delegados'' que puedesn ser acomodados de 
$9!$ maneras, ahora bien, las posibles arreglos de delegados de Irán y Rusia son $2!$.

Teniendo así $9! \cdot 2!$.

Los delegados de China y EE.UU. no pueden estar sentados juntos. Para calcularlo, primero se
calculará el número de formas de que los delegados de China y EE.UU. se sienten juntos y luego
restarlo.

Con los delegados de de Irán y Rusia sentados juntos, se tienen a 9 ``delegados''. Fuera de esto,
lo que queremos es encontrar de cuantas maneras China y EE.UU. delegados se pueden sentar juntos.
Siguiendo la misma idea, tomamos al delgado de China y EE.UU. como una sola entidad (``delegado'').
Ahora tenemos a 8 entidades que pueden ser acomodadas de $8!$ maneras. Los delegados de China y
EE.UU. pueden ser acomodados de $2!$ maneras posibles. También considerando a los delegados de
Rusia e Irán pueden ser acomodados de $2!$ maneras.

Finalmente restando el número de acomodos posibles de China y EE.UU sentados juntos y, Rusia e Irán
sentados juntos, del número de acomodos en el cual Rusia e Irán se sientas juntos son:
$$9! \cdot 2! - 8! \cdot 2! \cdot 2! = 564,480$$

\item Sean $A,B,C$ conjuntos. Prueba que
\begin{enumerate}[label=\alph*)]
    \item Conmutatividad.
    \begin{proof}
        Por demostrar $A \cup B = B \cup A$.
        
        Suponemos que $x \in A \cup B$. Entonces $x \in A$ o $x \in B$ o $x \in A \cap B$ lo
        cual escribimos $x \in B \cup A$. Por tanto $A \cup B = B \cup A$.
    \end{proof}

    \begin{proof}
        Por demostrar $A \cap B = B \cap A$.

        Suponemos que $x \in A \cap B$. Si $x$ está en ambos $A$ y $B$, entonces podemos decir que
        $x$ está en ambos $B$ y $A$ o $x \in B \cap A$. Por tanto $A \cap B = B \cap A$.
    \end{proof}

    \item Asociatividad.
    \begin{proof}
        Por demostrar que $A \cup (B \cup C) = (A \cup B) \cup C$.

        Suponemos que $x \in (A \cup B) \cup C$. Entonces se sigue que $x \in A \cup B$
        (lo que significa que $x \in A$ o $x \in B$) o $x \in C$. Por tanto podemos decir
        que $x \in A$ o $x \in B \cup C$ o en mejor $x \in A \cup (B \cup C)$. Por
        tanto $(A \cup B) \cup C \subseteq A \cup (A \cup C)$.

        Para mostrar que $(A \cup B) \cup C \supseteq A \cup (B \cup C)$ es análogo.
    \end{proof}

    \begin{proof}
        Por demostrar que $A \cap (B \cap C) = (A \cap B) \cap C$.

        Suponemos que $x \in (A \cap B) \cap C$. Entonces $x \in A \cap B$ y $x \in C$.
        Así que $x \in A$ y $x \in B \cap C$ o mejor $x \in A \cup (B \cup C)$. Por
        tanto $(A \cup B) \cup C \subseteq A \cup (A \cup C)$.

        Para mostrar que $(A \cap B) \cap C \supseteq A \cap (B \cap C)$ es análogo.
    \end{proof}

    \item Distributividad.
    
    \begin{proof}
        Por demostrar $A \cup (B \cap C) = (A \cup B) \cap (A \cup C)$.

        Primero se demostrará que $A \cup (B \cap C) \subset (A \cup B) \cap (A \cup C)$.

        Sea $x \in A \cup (B \cap C)$
        \begin{align*}
            &x \in A \quad \lor \quad x \in B \cap C\\
            &\implies x \in A \quad \lor \quad (x \in B \land x \in C)\\
            &\implies (x \in A \lor x \in B) \quad \land \quad (x \in A \lor x \in C)\\
            &\implies (x \in A \cup B) \quad \land \quad (x \in A \cup C)\\
            &\implies x \in (A \cup B) \cap (A \cup C) && \text{(1)}
        \end{align*}
        Por lo tanto $A \cup (B \cap C) \subset (A \cup B) \cap (A \cup C)$.

        La otra contención es $(A \cup B) \cap (A \cup C) \subset (A \cup C)$
        Sea $x \in (A \cup B) \cap (A \cup C)$
        \begin{align*}
            &x \in A \cup B \quad \land \quad x \in B \cup C\\
            &\implies (x \in A \quad \lor \quad x \in B) \quad \land \quad (x \in A \quad \lor \quad x \in C)\\
            &\implies x \in A \quad \lor \quad (x \in B \land x \in C)\\
            &\implies x \in A \quad \lor \quad (x \in B \cap C)\\
            &\implies x \in A (B \cup C) && \text{(2)}
        \end{align*}
        Por lo tanto $(A \cup B) \cap (A \cup C) \subset A \cup (A \cap C)$

        De $(1)$ y $(2)$ se tiene que $A \cup (B \cap C) = (A \cup B) \cap (A \cup C)$.
    \end{proof}

    \begin{proof}
        Por demostrar $A \cap (B \cup C) = (A \cap B) \cup (A \cap C)$.
        
        Análogo al anterior.
    \end{proof}


    \item $A \cup A = A$, $A \cap A = A$
    \begin{proof}
        TODO
    \end{proof}

    \item $A \subseteq A \cup B$, $A \cap B \subseteq A$
    \begin{proof}
        TODO
    \end{proof}

    \item $A \cup \varnothing = A$, $A \cap \varnothing = \varnothing$
    \begin{proof}
        TODO
    \end{proof}

    \item $A \cup (A \cap B) = A$, $A \cap (A \cup B) = A$
    \begin{proof}
        TODO
    \end{proof}

    \item Si $A \subseteq C$ y $B \subseteq C$ entonces $A \cup B \subseteq C$
    \begin{proof}
        TODO
    \end{proof}

    \item Si $C \subseteq A$ y $C \subseteq B$ entonces $C \subseteq A \cap B$
    \begin{proof}
        TODO
    \end{proof}

    \item Si $A \subset B$ y $B \subset C$ entonces $A \subset C$
    \begin{proof}
        TODO
    \end{proof}

    \item Si $A \subseteq B$ y $B \subseteq C$ entonces $A \subseteq C$
    \begin{proof}
        TODO
    \end{proof}

    \item Si $A \subset C$ y $B \subseteq C$ ¿Será cierto que $A \subset C$?
    \begin{proof}
        TODO
    \end{proof}

    \item Si $x \in A$ y $A \subseteq B$. ¿Será necesariamente que $x \in B$?
    \begin{proof}
       TODO 
    \end{proof}

    \item $A \setminus (B \cap C) = (A \setminus B) \cup (A \setminus C)$ 
    \begin{proof}
        Por contención:\\
        Por demostrar que $(A \setminus B) \cup (A \setminus C) \subseteq A \setminus (B \cap C)$
        Sea $x \in (A \setminus B) \cup (A \setminus C)$. Entonces $x \in A \setminus B$
        o $x \in A \setminus B$. Sin pérdida de generalidad, asumimos que $x \in A \setminus B$.
        Entonces $x \in A$ y $X \notin B$. Por tanto, $x \in A$ y $x \notin B \cap C$.
        Ergo $x \in A \setminus (B \cap C)$.

        Por demostrar que $A \setminus (B \cap C) \subset (A \subseteq B) \cup (A \setminus C)$.
        Sea $x \in A \setminus (B \cap C)$. Entonces $x \in A$ y $x \notin (B \cap C)$.
        Por tanto $x \in A$, y $x \notin B$ o $x \notin C$. Sin perdida de generalidad, asumimos
        que $x \notin B$. Entonces se tiene que $x \in A$ y $x \notin B$ y así $x \in A \setminus B$.
        Por tanto $x \in (A \setminus B) \cup (A \setminus C)$.

        Ergo $(A \setminus B) \cup (A \setminus C) = A \setminus (B \cap C)$.

    \end{proof}

\end{enumerate}

\item Prueba las leyes de De Morgan
\begin{enumerate}
    \item $(A \cup B)^c = A^c \cap B^c$
    \begin{proof}
        Sea $x \in (A \cup B)^c$. Entonces $x \notin A \cup B$. Así $x \notin A$ y $x \notin B$, es
        decir, $x \in A^c \cap B^c$ así $(A \cup B)^c \subseteq A^c \cap B^c$. Ahora sea
        $x \in A^c \cap B^c$. Entonces $x \in A^c$ y $x \in B^c$. Así $x \notin A$ y $x \notin B$,
        es decir, $x \notin A \cap B$ así $x \in (A \cap B)^c$. Por tanto $A^c \cap B^c \subseteq (A \cup B)^c$.
        Dado que $(A \cap B)^c \subseteq A^c \cup B^c$ y $(A \cap B)^c \supseteq A^c \cap B^c$.
        Ergo $(A \cup B)^c = A^c \cap B^c$.
    \end{proof}

    \item $(A \cap B)^c = A^c \cup B^c$
    \begin{proof}
        Sea $x \in (A \cap B)^c$. Entonces $x \notin A \cap B$. Así $x \notin A$ y $x \notin B$. 
        Por lo tanto $x \in A^c$ o $x \in B^c$, es
        decir, $x \in A^c \cup B^c$ así $(A \cap B)^c \subseteq A^c \cup B^c$. Ahora sea
        $x \in A^c \cup B^c$. Entonces $x \in A^c$ o $x \in B^c$ así $x \notin A$ y $x \notin B$,
        es decir, $x \notin A \cap B$ así $x \in (A \cap B)^c$. Por tanto $A^c \cup B^c \subseteq (A \cap B)^c$.
        Dado que $(A \cap B)^c \subseteq A^c \cup B^c$ y $(A \cap B)^c \supseteq A^c \cup B^c$.
        Ergo $(A \cap B)^c = A^c \cup B^c$.
    \end{proof}
\end{enumerate}

\item Prueba
\begin{enumerate}
    \item $A \setminus B = B^c \setminus A^c$
    \begin{proof}
        TODO
    \end{proof}

    \item $A \cap (B \setminus C) = (A \cap B) \setminus (A \cap C)$
    \begin{proof}
        TODO
    \end{proof}
\end{enumerate}

\item Prueba el Teorema del Binomio. Sean $x$, $y$ $\in \R$ y $n \in \N$. Entonces:
\[
    (x+y)^n = \sum_{i=0}^{n} \binom{n}{i} x^i y^{n-i}
\]
\begin{proof}
    Tomando el lado izquierdo de la igualdad, el producto de $(x+y)$ son $n$ copias de ella.
    \[
        (x+y)^n = \underbrace{(x+y)(x+y) \cdots (x+y)}_{n \text{ veces}}
    \]
    Ahora expandiendo esta expresión y multiplicando todos los términos. Como hay $n$ factores,
    cada monomío va a tener un variable en cada factor, y por tanto el grado va a ser de $n$.
    Más específico, cada monomío será de la forma $x^i y^{n-i}$ para alguna
    $i \in \{ 0,1,2, \ldots, n \}$ representando cuantas veces $x$ fue escogido.

    El coeficiente $x^k y^{n-i}$ es el número de maneras de escoger $x$ exactamente $i$ veces.
    Esto es equivalente a escoger un subconjunto de $k$ de los $n$ factores de los cuales se
    escogen $x$ (con $y$ siendo escogida del resto). Por definición, hay $\binom{n}{i}$
    subconjuntos, y por tanto el coeficiente de $x^k y^{n-i}$ es $\binom{n}{i}$.
    
    Ergo, sumando sobre todos los diferente monomíos se tiene que:
    $$(x+y)^n = \sum_{i=0}^{n} \binom{n}{i} x^i y^{n-i}$$
\end{proof}

\item Prueba $A \times (B \cup C) = (A \times B) \cup (A \times C)$
\begin{proof}
    \begin{align*}
        A \times (B \cup C)
        &= \{ (x,y): (x \in A) \land (y \in B \cup C)\} && \text{Def. produto cartesiano}\\
        &= \{ (x,y): (x \in A) \land (y \in B) \lor (y \cap C)\} && \text{Def. de $\cup$}\\
        &= \{ (x,y): (x \in A) \land (x \in A) \land (y \in B) \lor (y \in C)\} && P = P \lor P\\
        &= \{ (x,y): ((x \in A) \land (y \in B)) \lor ((x \in A) \land (y \in C)) \} && \text{Reacomodo}\\
        &= \{ (x,y): (x \in A) \land (y \in B) \} \cup \{ (x,y): (x \in A) \land (y \in C) \} && \text{Def. de $\cup$}\\
        &= (A \times B) \cup (A \times C) && \text{Def. de $\times$} 
    \end{align*}
\end{proof}

\item Si $A = B \cap C$. Determina si es cierto o no
\begin{enumerate}
    \item $A \times A = (B \times B) \cap (C \times C)$\\
    Cierto.

    \item $A \times A = (B \times C) \cap (C \times B)$\\
    Falso.
\end{enumerate}


\item Prueba que una de las dos identidades siguientes es siempre correcta y la otra algunas veces
es falsa:
\begin{enumerate}
    \item $A \setminus (B \setminus C) = (A \setminus B) \cup C$
    
    TODO

    \item $A \setminus (B \cup C) = (A \setminus B) \setminus C$
    
    TODO
\end{enumerate}

\item Sea $\Lambda$ una clase de conjuntos. Prueba que
$$B \setminus \bigcup_{A \in \Lambda} A = \bigcap_{A \in \Lambda} (B \setminus A)$$
\begin{proof}
    \begin{align*}
        & x \in B \setminus \bigcup_{A \in \Lambda} A\\
        &\iff x \in B \quad \land \quad x \notin \bigcup_{A \in \Lambda} A\\
        &\iff x \in B \quad \land \quad x \notin A, \forall A \in \Lambda\\
        &\iff x \in B \setminus A, \quad \forall A \in \Lambda\\
        &\iff x \in \bigcap_{A \in \Lambda} (B \setminus A)\\
        &\therefore B \setminus \bigcup_{A \in \Lambda} A = \bigcap_{A \in \Lambda} (B \setminus A)
    \end{align*}
\end{proof}

\iffalse
\item Prueba el principio de inclusión-exclusión.
$$P(\bigcup_{i=1}^{n} A_i) = \sum_{i=1}^{n} P(A_i) - \sum_{1 \leq i < j \leq n } P(A_i \cap A_j)
+ \ldots + (-1)^{n+1} P(A_1 \cap \ldots \cap A_n)$$
\begin{proof}
    YA
\end{proof}
\fi

\begin{figure}[H]
    \centering
    \includegraphics[scale=0.9]{gato.png}
    \caption{No la acabé toda la tarea :(}
\end{figure}
\end{enumerate}

%%%%%%%%%%%%%%%%%%%%%%%%%%%%%%%%%%%%%%%%%%%%%%%%%%%%%%%%%%%%%%%%%%%%%%%%%%%%%%%%%%%%%%%%%


%%%%%%%%%%%%%%%%%%%%%%%%%%%%%%%%%%%%%%%%%%%%%%%%%%%%%%%%%%%%%%%%%%%%%%%%%%%%%%%%%%%%%%%%%

%%%%%%%%%%%%%%%%%%%%%%%%%%%%%%%%%%%%%%%%%%%%%%%%%%%%%%%%%%%%%%%%%%%%%%%%%%%%%%%%%%%%%%%%%

\end{document}