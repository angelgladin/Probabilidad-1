%%%
 %
 % Copyright (C) 2019 Ángel Iván Gladín García
 %
 % This program is free software: you can redistribute it and/or modify
 % it under the terms of the GNU General Public License as published by
 % the Free Software Foundation, either version 3 of the License, or
 % (at your option) any later version.
 %
 % This program is distributed in the hope that it will be useful,
 % but WITHOUT ANY WARRANTY; without even the implied warranty of
 % MERCHANTABILITY or FITNESS FOR A PARTICULAR PURPOSE.  See the
 % GNU General Public License for more details.
 %
 % You should have received a copy of the GNU General Public License
 % along with this program.  If not, see <http://www.gnu.org/licenses/>.
%%%

%%%%%%%%%%%%%%%%%%%%%%%%%%%%%%%%%%%%%%%%%%%%%%%%%%%%%%%%%%%%%%%%%%%%%%%%%%%%%%%%%%%%%%%%%
\documentclass[11pt,letterpaper]{report}
\usepackage[margin=.7in]{geometry}
\usepackage[utf8]{inputenc}
\usepackage[spanish]{babel}
\decimalpoint

\usepackage{listings}
\usepackage{color}
\usepackage{graphicx}
\usepackage{enumerate}
\usepackage{enumitem}
\usepackage{float}

\usepackage{longtable}
\usepackage{hyperref}
\usepackage{commath}

\usepackage{bbm}
\usepackage{dsfont}
\usepackage{mathrsfs}
\usepackage{amsmath,amsthm,amssymb}
\usepackage{mathtools}
\usepackage{longtable}

\usepackage{tikz}
\usetikzlibrary{trees}
\usepackage{verbatim}

% Set the overall layout of the tree
\tikzstyle{level 1}=[level distance=3.5cm, sibling distance=3.5cm]
\tikzstyle{level 2}=[level distance=3.5cm, sibling distance=2cm]

% Define styles for bags and leafs
\tikzstyle{bag} = [text width=4em, text centered]
\tikzstyle{end} = [circle, minimum width=3pt,fill, inner sep=0pt]
%%%%%%%%%%%%%%%%%%%%%%%%%%%%%%%%%%%%%%%%%%%%%%%%%%%%%%%%%%%%%%%%%%%%%%%%%%%%%%%%%%%%%%%%%%%%%%%%5

\usepackage{import}

\usepackage[utf8]{inputenc}

\usepackage{listings}
\usepackage{color}

\definecolor{codegreen}{rgb}{0,0.6,0}
\definecolor{codegray}{rgb}{0.5,0.5,0.5}
\definecolor{codepurple}{rgb}{0.58,0,0.82}
\definecolor{backcolour}{rgb}{0.95,0.95,0.92}

\lstdefinestyle{mystyle}{
    backgroundcolor=\color{backcolour},   
    commentstyle=\color{codegreen},
    keywordstyle=\color{magenta},
    numberstyle=\tiny\color{codegray},
    stringstyle=\color{codepurple},
    basicstyle=\footnotesize,
    breakatwhitespace=false,         
    breaklines=true,                 
    captionpos=b,                    
    keepspaces=true,                 
    numbers=left,                    
    numbersep=5pt,                  
    showspaces=false,                
    showstringspaces=false,
    showtabs=false,                  
    tabsize=2
}

\lstset{style=mystyle}
%%%%%%%%%%%%%%%%%%%%%%%%%%%%%%%%%%%%%%%%%%%%%%%%%%%%%%%%%%%%%%%%%%%%%%%%%%%%%%%%%%%%%%%%%


%%%%%%%%%%%%%%%%%%%%%%%%%%%%%%%%%%%%%%%%%%%%%%%%%%%%%%%%%%%%%%%%%%%%%%%%%%%%%%%%%%%%%%%%%
\newcommand{\Z}{\mathbb{Z}}
\newcommand{\N}{\mathbb{N}}
\newcommand{\Q}{\mathbb{Q}}
\newcommand{\R}{\mathbb{R}}
\newcommand{\Pro}{\mathds{P}}
\newcommand{\Oh}{\mathcal{O}} %% Notacion "O"
\newcommand{\lra}{\longrightarrow}
\newcommand{\ra}{\rightarrow}
\newcommand{\ord}{\text{ord}}
\newcommand{\sol}{\textbf{\underline{Solución}: }} %% Solucion
\newcommand{\af}{\textbf{\underline{Afirmación}: }}
\newcommand{\cej}{\textbf{\underline{Contraejemplo}: }}

%%%%%%%%%%%%%%%%%%%%%%%%%%%%%%%%%%%%%%%%%%%%%%%%%%%%%%%%%%%%%%%%%%%%%%%%%%%%%%%%%%%%%%%%%

\begin{document}

%%%%%%%%%%%%%%%%%%%%%%%%%%%%%%%%%%%%%%%%%%%%%%%%%%%%%%%%%%%%%%%%%%%%%%%%%%%%%%%%%%%%%%%%%
\title{
        Universidad Nacional Autónoma de México\\
        Facultad de Ciencias\\
        Probabilidad I\\
    \vspace{1cm}
    \large
        \textbf{Tarea 4}\\
        \textbf{Variables Aleatorias Discretas I}
}
\author{
    Ángel Iván Gladín García\\
    No. cuenta: 313112470\\
    \texttt{angelgladin@ciencias.unam.mx}
}
\date{-1 de octubre 2019}
\maketitle
%%%%%%%%%%%%%%%%%%%%%%%%%%%%%%%%%%%%%%%%%%%%%%%%%%%%%%%%%%%%%%%%%%%%%%%%%%%%%%%%%%%%%%%%%

%%%%%%%%%%%%%%%%%%%%%%%%%%%%%%%%%%%%%%%%%%%%%%%%%%%%%%%%%%%%%%%%%%%%%%%%%%%%%%%%%%%%%%%%%
\newtheorem{theorem}{Teorema}
\newtheorem{example}{Ejemplo}
\newtheorem{corollary}{Corolario}
\newtheorem{lemma}{Lemma}
\newtheorem{definition}{Definicion}
\newtheorem{prop}{Proposicion}
%%%%%%%%%%%%%%%%%%%%%%%%%%%%%%%%%%%%%%%%%%%%%%%%%%%%%%%%%%%%%%%%%%%%%%%%%%%%%%%%%%%%%%%%%

%%%%%%%%%%%%%%%%%%%%%%%%%%%%%%%%%%%%%%%%%%%%%%%%%%%%%%%%%%%%%%%%%%%%%%%%%%%%%%%%%%%%%%%%%
\begin{enumerate}

%%%%%%%%%%%%%%% (1)
\item Supón que la función de distribución acumulada $X$ está dada por:
\[
    F(b) =
        \begin{cases}
            0   & b < 0\\
            \frac{b}{4}    & 0 \leq b < 1\\
            \frac{1}{2} + \frac{b-1}{4}    & 1 \leq b < 2\\
            \frac{11}{12}    & 2 \leq b < 3\\
            1   & b \geq 3
        \end{cases}
\]
\begin{itemize}
    \item Obtén $P(X = i)$ si $i=1,2,3$
    
    \sol

    \item Calcula $P(\frac{1}{2} < X < \frac{3}{2})$
    
    \sol
\end{itemize}

%%%%%%%%%%%%%%% (2)
\item Supón que la función de masa de probabilidad de la Variable Aleatoria $X$ está dada por
\[
    p_X(i) = ci \text{ donde } i = 1,2,3,4,5,6
\]
\begin{itemize}
    \item Obtén el valor de $c$.
    
    \sol

    
    \item Obtén la probabilidad de que $X$ tome algún valor ``Par''.
    
    \sol
\end{itemize}

%%%%%%%%%%%%%%% (3)
\item Sea $X$ una Variable Aleatoria con función de masa de probabilidad dada por:
\[
    P(X=-1)= \frac{1}{2},\quad P(X=0)=\frac{1}{3},\quad P(X=1)=\frac{1}{6}
    \]
    Encuentra el valor de $\mathds{E}(|X|)$ de las siguientes maneras:
    \begin{itemize}
        \item Obtén la f.m.p de la Variable Aleatoria $Y = |X|$, usa este resultado para obtener $\mathds{E}|X|$.
        
        \sol
        
        \item Usa la Ley del ``Estadistico Inconsiente'' con la función $g(x) = |x|$.
        
        \sol
    \end{itemize}
    
%%%%%%%%%%%%%%% (4)
\item Natalia coloca 5 cajitas cerradas sobre una mesa durante una reunión familiar. Tres de ellas
contienen algún obsequio mientras que las otras dos cajas no tienen nada. Si alguno de sus sobrinos
pequeños comienza a abrir las cajas. Sea $X$ la variable aleatoria que denota:
\begin{center}
    $X$ = El número de cajas que abre su sobrino hasta que obtiene el primer regalo.
\end{center}

\begin{itemize}
    \item Obtén la función de masa de probabilidad asociada a $X$
    
    \sol

    \item Obtén $\mathds{E}(X)$.
    
    \sol

    \item Obtén $Var(X)$.
    
    \sol
\end{itemize}

%%%%%%%%%%%%%%% (5)
\item Una ``Baraja Inglesa'' tiene 52 cartas. Si un paquete de cartas de este juego es barajeado y
se van seleccionando cartas una a una hasta el instante que un A's aparece. Obtén el número esperado
de cartas que se tuvieron que haber volteado una a una, hasta el momento en que un A's es seleccionado.

\sol

%%%%%%%%%%%%%%% (6)
\item En un sorteo de la Loteria Nacional se venden 2,000,000 de raspaditos cuyo valor es de \$10
pesos cada uno. Si 4000 de ellos tienen un premio de un valor de \$300 pesos. 500 boletos tienen un
premio de \$8000 mientras que solo un boleto tiene un premio de \$10,000,000. Si ningún boleto tiene
mas de un premio. ¿Cuál es el valor esperado del premio que ganara un jugador que compra un boleto
de este sorteo?

\sol

%%%%%%%%%%%%%%% (7)
\item Sea $X$ una Variable Aleatoria con función de masa de probabilidad: 
\[
    P(X = 1) = p = 1 - P(X = -1)
\]
Obtén $c \not= 1$ tal que $\mathds{E}(cx)=1$

\sol


%%%%%%%%%%%%%%% (8)
%%%%%%%%%%%%%%% (9)
%%%%%%%%%%%%%%% (10)
%%%%%%%%%%%%%%% (11)
%%%%%%%%%%%%%%% (12)
%%%%%%%%%%%%%%% (13)
%%%%%%%%%%%%%%% (14)











\begin{figure}[H]
        \centering
        \includegraphics[scale=0.25]{gato.png}
        \caption{Mi cara cuando vi la tarea por primera vez.}
\end{figure}

\end{enumerate}
%%%%%%%%%%%%%%%%%%%%%%%%%%%%%%%%%%%%%%%%%%%%%%%%%%%%%%%%%%%%%%%%%%%%%%%%%%%%%%%%%%%%%%%%%


%%%%%%%%%%%%%%%%%%%%%%%%%%%%%%%%%%%%%%%%%%%%%%%%%%%%%%%%%%%%%%%%%%%%%%%%%%%%%%%%%%%%%%%%%

%%%%%%%%%%%%%%%%%%%%%%%%%%%%%%%%%%%%%%%%%%%%%%%%%%%%%%%%%%%%%%%%%%%%%%%%%%%%%%%%%%%%%%%%%

\end{document}