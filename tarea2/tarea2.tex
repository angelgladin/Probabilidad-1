%%%
 %
 % Copyright (C) 2019 Ángel Iván Gladín García
 %
 % This program is free software: you can redistribute it and/or modify
 % it under the terms of the GNU General Public License as published by
 % the Free Software Foundation, either version 3 of the License, or
 % (at your option) any later version.
 %
 % This program is distributed in the hope that it will be useful,
 % but WITHOUT ANY WARRANTY; without even the implied warranty of
 % MERCHANTABILITY or FITNESS FOR A PARTICULAR PURPOSE.  See the
 % GNU General Public License for more details.
 %
 % You should have received a copy of the GNU General Public License
 % along with this program.  If not, see <http://www.gnu.org/licenses/>.
%%%

%%%%%%%%%%%%%%%%%%%%%%%%%%%%%%%%%%%%%%%%%%%%%%%%%%%%%%%%%%%%%%%%%%%%%%%%%%%%%%%%%%%%%%%%%
\documentclass[11pt,letterpaper]{report}
\usepackage[margin=1in]{geometry}
\usepackage[utf8]{inputenc}
\usepackage[spanish]{babel}

\usepackage{listings}
\usepackage{color}
\usepackage{graphicx}
\usepackage{enumerate}
\usepackage{enumitem}
\usepackage{float}

\usepackage{longtable}
\usepackage{hyperref}
\usepackage{commath}

\usepackage{bbm}
\usepackage{dsfont}
\usepackage{mathrsfs}
\usepackage{amsmath,amsthm,amssymb}
\usepackage{mathtools}
\usepackage{longtable}
%%%%%%%%%%%%%%%%%%%%%%%%%%%%%%%%%%%%%%%%%%%%%%%%%%%%%%%%%%%%%%%%%%%%%%%%%%%%%%%%%%%%%%%%%%%%%%%%5

\usepackage{import}

\usepackage[utf8]{inputenc}
 
\usepackage{listings}
\usepackage{color}
 
\definecolor{codegreen}{rgb}{0,0.6,0}
\definecolor{codegray}{rgb}{0.5,0.5,0.5}
\definecolor{codepurple}{rgb}{0.58,0,0.82}
\definecolor{backcolour}{rgb}{0.95,0.95,0.92}
 
\lstdefinestyle{mystyle}{
    backgroundcolor=\color{backcolour},   
    commentstyle=\color{codegreen},
    keywordstyle=\color{magenta},
    numberstyle=\tiny\color{codegray},
    stringstyle=\color{codepurple},
    basicstyle=\footnotesize,
    breakatwhitespace=false,         
    breaklines=true,                 
    captionpos=b,                    
    keepspaces=true,                 
    numbers=left,                    
    numbersep=5pt,                  
    showspaces=false,                
    showstringspaces=false,
    showtabs=false,                  
    tabsize=2
}
 
\lstset{style=mystyle}
%%%%%%%%%%%%%%%%%%%%%%%%%%%%%%%%%%%%%%%%%%%%%%%%%%%%%%%%%%%%%%%%%%%%%%%%%%%%%%%%%%%%%%%%%


%%%%%%%%%%%%%%%%%%%%%%%%%%%%%%%%%%%%%%%%%%%%%%%%%%%%%%%%%%%%%%%%%%%%%%%%%%%%%%%%%%%%%%%%%
\newcommand{\Z}{\mathbb{Z}}
\newcommand{\N}{\mathbb{N}}
\newcommand{\Q}{\mathbb{Q}}
\newcommand{\R}{\mathbb{R}}
\newcommand{\Oh}{\mathcal{O}} %% Notacion "O"
\newcommand{\lra}{\longrightarrow}
\newcommand{\ra}{\rightarrow}
\newcommand{\ord}{\text{ord}}
\newcommand{\sol}{\textbf{\underline{Solución}: }} %% Solucion

%%%%%%%%%%%%%%%%%%%%%%%%%%%%%%%%%%%%%%%%%%%%%%%%%%%%%%%%%%%%%%%%%%%%%%%%%%%%%%%%%%%%%%%%%

\begin{document}

%%%%%%%%%%%%%%%%%%%%%%%%%%%%%%%%%%%%%%%%%%%%%%%%%%%%%%%%%%%%%%%%%%%%%%%%%%%%%%%%%%%%%%%%%
\title{
        Universidad Nacional Autónoma de México\\
        Facultad de Ciencias\\
        Probabilidad I\\
    \vspace{1cm}
    \large
        \textbf{Tarea 2}\\
        \textbf{Propiedades Básicas de la Probabilidad}
}
\author{
    Ángel Iván Gladín García\\
    No. cuenta: 313112470\\
    \texttt{angelgladin@ciencias.unam.mx}
}
\date{-1 de Septiembre 2019}
\maketitle
%%%%%%%%%%%%%%%%%%%%%%%%%%%%%%%%%%%%%%%%%%%%%%%%%%%%%%%%%%%%%%%%%%%%%%%%%%%%%%%%%%%%%%%%%

%%%%%%%%%%%%%%%%%%%%%%%%%%%%%%%%%%%%%%%%%%%%%%%%%%%%%%%%%%%%%%%%%%%%%%%%%%%%%%%%%%%%%%%%%
\newtheorem{theorem}{Teorema}
\newtheorem{example}{Ejemplo}
\newtheorem{corollary}{Corolario}
\newtheorem{lemma}{Lemma}
\newtheorem{definition}{Definicion}
\newtheorem{prop}{Proposicion}
%%%%%%%%%%%%%%%%%%%%%%%%%%%%%%%%%%%%%%%%%%%%%%%%%%%%%%%%%%%%%%%%%%%%%%%%%%%%%%%%%%%%%%%%%

%%%%%%%%%%%%%%%%%%%%%%%%%%%%%%%%%%%%%%%%%%%%%%%%%%%%%%%%%%%%%%%%%%%%%%%%%%%%%%%%%%%%%%%%%
\begin{enumerate}

\item Imaginemos el siguiente experimento aleatorio. ``Se Lanza un dado continuamente hasta que
caiga un 6 y en ese momento dejamos de lanzar el dado''.

\begin{itemize}
    \item Describe el Espacio Muestral asociado a este Experimento Aleatorio
    

    \item Sea $E_n$ el evento que denota: El número de lanzamientos necesarios hasta completar el
    experimento. ¿Cuáles puntos del espacio muestral están contenidos en $E_n$?


    \item ¿Qué representa $(\cup_{n=1}^{\infty} E_n)^c$ en términos del experimento aleatorio?
    
    
\end{itemize}
    
    
\item Tres individuos $A$, $B$ , $C$ se turnan para lanzar una moneda. Al primero que le salga
``Águila'' gana. El espacio muestral asociado a este experimento puede ser definido como:
\[
    S =
    \begin{cases}
        1, 01, 001, 0001, \ldots,\\
        0000\ldots
    \end{cases}
\]

\begin{itemize}
    \item Interpreta el Espacio Muestral
    \item Define los siguientes eventos en terminos de $S$

    \begin{enumerate}[label=\alph*)]
        \item $A$ gana = $A$
        \item $B$ gana = $B$
        \item $(A \cup B)^c$
        
        Asume que $A$ lanza primero la moneda, denoganar lanza $B$, de no ganarlanza $C$ y en
        caso de no ganar nadie se repite el proceso.    
    \end{enumerate}

\end{itemize}

\item Si al jugar \textbf{Poker} asumimos que las $\binom{52}{2}$ manos son igualmente probables.
¿Cuál es la probabilidad de que nos hayan repartido:

\begin{itemize}
    \item una Flor? \textbf{Decimos que una mano de poker es una flor si las 5 cartas de la mano son
    del mismo palo} %$\spadesuit, \vardiamond, \clubsuit, \varheart$.
    
    
    \item un Par?
    

    \item dos Pares?
    
    \item una Tercia?
    
    \item un Poker? \textbf{Tenemos un Poker cuando la mano que nos tocaron contiene 4 cartas con la
    misma denominación y de diferente palo}
    
\end{itemize}

\item Los coeficientes de la ecuación cuadrática $x^2 + bx + c$ están determinados al lanzar un par
de dados, el resultado del primer dado determina el valor de $b$ y el del segundo dado el valor de
$c$. Obtén la probabilidad de que la ecuación tenga soluciones reales.

\item Dos números $m$ y $n$ son llamados "primos relativos" si el 1 es el único divisor común
positivo entre ambos. Así por ejemplo el 8 y el 5 son primos relativos, mientras el 8 y 6 no lo son.
Si un nuúmero es seleccionado al azar del conjunto $\{ 1, 2, 3,\ldots, 63 \}$ . Obtén la
probabilidad de que este número sea primo relativo con el 63.

\item Una clase de Biología tiene 33 alumnos inscritos. Si 17 de ellos sacaron 10's en el primer
examen parcial y 14 de ellos sacaron 10's en el segundo examen parcial mientras 11 alumnos no
obtuvieron un 10's en ninguno de estos dos exámenes. ¿Cuál es la probabilidad de que un estudiante
seleccionado al azar de esta clase haya sacado 10's en ambos exámenes?

\item Una terapeuta acomodara al azar a 5 matrimonios sobre una fila para realizar una actividad.
Obtén la probabilidad de que no haya algun esposo sentado junto a su mujer sobre la fila.


    
\end{enumerate}


%%%%%%%%%%%%%%%%%%%%%%%%%%%%%%%%%%%%%%%%%%%%%%%%%%%%%%%%%%%%%%%%%%%%%%%%%%%%%%%%%%%%%%%%%


%%%%%%%%%%%%%%%%%%%%%%%%%%%%%%%%%%%%%%%%%%%%%%%%%%%%%%%%%%%%%%%%%%%%%%%%%%%%%%%%%%%%%%%%%

%%%%%%%%%%%%%%%%%%%%%%%%%%%%%%%%%%%%%%%%%%%%%%%%%%%%%%%%%%%%%%%%%%%%%%%%%%%%%%%%%%%%%%%%%

\end{document}