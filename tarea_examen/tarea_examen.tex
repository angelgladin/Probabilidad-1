%%%
 %
 % Copyright (C) 2019 Ángel Iván Gladín García
 %
 % This program is free software: you can redistribute it and/or modify
 % it under the terms of the GNU General Public License as published by
 % the Free Software Foundation, either version 3 of the License, or
 % (at your option) any later version.
 %
 % This program is distributed in the hope that it will be useful,
 % but WITHOUT ANY WARRANTY; without even the implied warranty of
 % MERCHANTABILITY or FITNESS FOR A PARTICULAR PURPOSE.  See the
 % GNU General Public License for more details.
 %
 % You should have received a copy of the GNU General Public License
 % along with this program.  If not, see <http://www.gnu.org/licenses/>.
%%%

%%%%%%%%%%%%%%%%%%%%%%%%%%%%%%%%%%%%%%%%%%%%%%%%%%%%%%%%%%%%%%%%%%%%%%%%%%%%%%%%%%%%%%%%%
\documentclass[11pt,letterpaper]{report}
\usepackage[margin=.75in]{geometry}
\usepackage[utf8]{inputenc}
\usepackage[spanish]{babel}
\decimalpoint

\usepackage{listings}
\usepackage{color}
\usepackage{graphicx}
\usepackage{enumerate}
\usepackage{enumitem}
\usepackage{float}

\usepackage{longtable}
\usepackage{hyperref}
\usepackage{commath}

\usepackage{bbm}
\usepackage{dsfont}
\usepackage{mathrsfs}
\usepackage{amsmath,amsthm,amssymb}
\usepackage{mathtools}
\usepackage{longtable}
%%%%%%%%%%%%%%%%%%%%%%%%%%%%%%%%%%%%%%%%%%%%%%%%%%%%%%%%%%%%%%%%%%%%%%%%%%%%%%%%%%%%%%%%%%%%%%%%5

\usepackage{import}

\usepackage[utf8]{inputenc}

\usepackage{listings}
\usepackage{color}

\definecolor{codegreen}{rgb}{0,0.6,0}
\definecolor{codegray}{rgb}{0.5,0.5,0.5}
\definecolor{codepurple}{rgb}{0.58,0,0.82}
\definecolor{backcolour}{rgb}{0.95,0.95,0.92}

\lstdefinestyle{mystyle}{
    backgroundcolor=\color{backcolour},   
    commentstyle=\color{codegreen},
    keywordstyle=\color{magenta},
    numberstyle=\tiny\color{codegray},
    stringstyle=\color{codepurple},
    basicstyle=\footnotesize,
    breakatwhitespace=false,         
    breaklines=true,                 
    captionpos=b,                    
    keepspaces=true,                 
    numbers=left,                    
    numbersep=5pt,                  
    showspaces=false,                
    showstringspaces=false,
    showtabs=false,                  
    tabsize=2
}

\lstset{style=mystyle}
%%%%%%%%%%%%%%%%%%%%%%%%%%%%%%%%%%%%%%%%%%%%%%%%%%%%%%%%%%%%%%%%%%%%%%%%%%%%%%%%%%%%%%%%%


%%%%%%%%%%%%%%%%%%%%%%%%%%%%%%%%%%%%%%%%%%%%%%%%%%%%%%%%%%%%%%%%%%%%%%%%%%%%%%%%%%%%%%%%%
\newcommand{\Z}{\mathbb{Z}}
\newcommand{\N}{\mathbb{N}}
\newcommand{\Q}{\mathbb{Q}}
\newcommand{\R}{\mathbb{R}}
\newcommand{\Pro}{\mathds{P}}
\newcommand{\Oh}{\mathcal{O}} %% Notacion "O"
\newcommand{\lra}{\longrightarrow}
\newcommand{\ra}{\rightarrow}
\newcommand{\ord}{\text{ord}}
\newcommand{\sol}{\textbf{\underline{Solución}: }} %% Solucion

%%%%%%%%%%%%%%%%%%%%%%%%%%%%%%%%%%%%%%%%%%%%%%%%%%%%%%%%%%%%%%%%%%%%%%%%%%%%%%%%%%%%%%%%%

\begin{document}

%%%%%%%%%%%%%%%%%%%%%%%%%%%%%%%%%%%%%%%%%%%%%%%%%%%%%%%%%%%%%%%%%%%%%%%%%%%%%%%%%%%%%%%%%
\title{
        Universidad Nacional Autónoma de México\\
        Facultad de Ciencias\\
        Probabilidad I\\
    \vspace{1cm}
    \large
        \textbf{Tarea-Examen}\\
        \textbf{Probabilidad Condicional e Independencia de Eventoes}
}
\author{
    Ángel Iván Gladín García\\
    No. cuenta: 313112470\\
    \texttt{angelgladin@ciencias.unam.mx}
}
\date{24 de Septiembre 2019}
\maketitle
%%%%%%%%%%%%%%%%%%%%%%%%%%%%%%%%%%%%%%%%%%%%%%%%%%%%%%%%%%%%%%%%%%%%%%%%%%%%%%%%%%%%%%%%%

%%%%%%%%%%%%%%%%%%%%%%%%%%%%%%%%%%%%%%%%%%%%%%%%%%%%%%%%%%%%%%%%%%%%%%%%%%%%%%%%%%%%%%%%%
\newtheorem{theorem}{Teorema}
\newtheorem{example}{Ejemplo}
\newtheorem{corollary}{Corolario}
\newtheorem{lemma}{Lemma}
\newtheorem{definition}{Definicion}
\newtheorem{prop}{Proposicion}
%%%%%%%%%%%%%%%%%%%%%%%%%%%%%%%%%%%%%%%%%%%%%%%%%%%%%%%%%%%%%%%%%%%%%%%%%%%%%%%%%%%%%%%%%

%%%%%%%%%%%%%%%%%%%%%%%%%%%%%%%%%%%%%%%%%%%%%%%%%%%%%%%%%%%%%%%%%%%%%%%%%%%%%%%%%%%%%%%%%
\begin{enumerate}

%%%%%%%%%%%%%%% (1)
\item Prueba los siguientes resultados:
\begin{itemize}
    \item Sean $E, F \in S$ eventos mutuamente excluyentes entonces:
    \[
        \Pro(E|E \cup F) = \frac{\Pro(E)}{\Pro(E) + \Pro(F)}
    \]
    \sol \begin{proof}
        
    \end{proof}
    
    \item Sea $\{ E_i \}_{i=0}^\infty \subseteq S$ sucesión de eventos mutuamente excluyentes
    entonces:
    \[
        \Pro(E_j | \cup_{i=1}^{\infty} E_i) = \frac{\Pro(E_j)}
            {\sum_{i=1}^{\infty} \Pro(E_i)}
    \]
    \sol \begin{proof}
        
    \end{proof}
\end{itemize}

%%%%%%%%%%%%%%% (2)
\item Sean $E_1, E_2, \ldots , E_n \in S$ eventos independientes. Prueba que:
\[
    \Pro(E_1 \cup E_2 \cup \ldots \cup E_n) = 1 - \prod_{i=1}^{\infty} [1 - \Pro(E_i)]
\]
\sol \begin{proof}
    
\end{proof}

%%%%%%%%%%%%%%% (3)
\item Una serie Experimentos Aleatorios Independientes cuyos posibles resultados pueden ser
clasificados como \emph{Exitos} ó \emph{Fracasos}. Donde ocurre un éxito con probabilidad $p$ y un
fracaso con probabilidad $1-p$ en cada ensayo son llamados \emph{Ensayos de Bernoulli}
$p \in [0, 1]$. Si $P_n$ denota la probabilidad de que si se realizan $n$ ensayos de bernoulli
ocurran un número par de éxitos (0 será considerado como un número par). Prueba que:
\[
    P_n = p(1 - P_{n-1})+(1-p)P_{n-1} \quad \text{si } n \geq 1
\]
Usa esta fórmula para demostrar (por inducción) que:
\[
    P_n = \frac{1+ (1-2p)^n}{2}
\]
\sol \begin{proof}
    
\end{proof}

%%%%%%%%%%%%%%% (4)
\item Imagina que tenemos una sucesión $\{ a_i \}_{i=1}^{\infty} \subset \R$ tal que
$0 \leq a_i \leq 1$ para toda $i \geq 1$.

Demuestra lo siguiente:

\[
    \sum_{i=1}^{\infty} a_i (\prod_{j=1}^{i-1} (i - a_i)) +
        \prod_{i=1}^{\infty} (1 - a_i) = 1
\]

\textit{Hint:} Supón que se realizan una infinidad de ``Volados''. Sea $a_i$ la probabilidad de que
en el i-ésimo lanzamiento cae por primera vez Águila.

\sol \begin{proof}
    
\end{proof}

%%%%%%%%%%%%%%% (5)
\item Valería tiene 2 bolsas de dulces. La primera de ellas contiene 2 paquetes de Kranky's y 3
paquetes de Panditas. La segunda bolsa contiene 4 paquetes de Kranky's y 2 paquetes de Panditas.
Si ella elige al azar una bolsa y elige un paquete de dulces. ¿Cuál es la probabilidad de que
Valeria haya elegido un paquete de Panditas?.

Muestra el Diagrama de Arbol asociado a este problema.

\sol

%%%%%%%%%%%%%%% (6)
\item Se realizó una encuesta en la Facultad de Ciencias. Se encontró que la probabilidad de
seleccionar una persona al azar y que esta sea fumadora es del 20\%. La probabilidad de que al
menos la persona tenga un familiar que fume es del 30\%. Además si esta persona al menos tiene un
familiar fumador la probabilidad de que ella misma fume se eleva hasta 35\%. Obtén la probabilidad
de que la persona elegida sea fumadora pero que no tenga ningún familiar que fume.

\sol

%%%%%%%%%%%%%%% (7)
\item De acuerdo con un estudio realizado por la Secretaría de Salud. La probabilidad de que una
mujer de entre 50 y 59 años sea diagnosticada con Cáncer de Mama es de 2.38\%. Se sabe que si una
mujer bajo estas condiciones se realiza una Mamografía esta prueba tiene un nivel de sensitividad
tumoral del 85\% para mujeres de este rango de edad. Si el nivel de confianza de esta prueba es del
95\%. Esto es la tasa de Falsos-Negativos es del 15\% y la de Falsos-Positivos es de 5\%. Si una
mujer en este rango de edad se realiza una Mamografía y esta resulta positiva para Cáncer. ¿Cuál
es la probabilidad de que ella realmente tenga la enfermedad?

\sol

%%%%%%%%%%%%%%% (8)
\item  Un Polígrafo (Detector de Mentiras) decimos que es 90\% confiable en el siguiente sentido: Hay
un 90\% de chances de que una persona que está diciendo la verdad pase la prueba del polígrafo y
hay un 90\% de chances que una persona que este mintiendo falle la prueba del polígrafo.
\begin{itemize}
    \item En la CDMX se estima que el 5\% de las personas miente. Si una persona seleccionada al
    azar en la ciudad realiza la prueba del polígrafo la cual dice que persona está mintiendo.
    ¿Cuál es la probabilidad de que esta persona verdaderamente esté mintiendo?
    
    \sol 

    \item ¿Qué tan confiable debería ser la prueba del poligrafo para que la probabilidad de que una
    persona que verdaderamente esté mintiendo dado que la prueba del poligrafo resulto positiva sea
    de al menos 80\%?
    
    \sol
\end{itemize}

%%%%%%%%%%%%%%% (9)
\item Una persona en Los Angeles California observa que un Taxi atropella a una persona. Si en L.A.
el 95\% de los Taxis son amarillos y el 5\% son azules. Un perito de la L.A.P.D.(Policía) cree que
una persona que atestigua un accidente es 80\% confiable. Esto es que la persona identifica el color
del Taxi el 80\% de las veces. ¿Cuál es la probabilidad de que el Taxi que participó en este
accidente haya sido azul?

\sol

%%%%%%%%%%%%%%% (10)
\item Imagina que un amigo tuyo tiene 3 dados. Uno de ellos es standard, el segundo de ellos solo
tiene 5's en cada uno de sus lados y el otro tiene 5's en tres caras y 4's en las 3 caras restantes.
Si un dado de ellos es seleccionado al azar y se lanza y cae un 5. ¿Cual es la probabilidad de que
el dado elegido haya sido el dado standard?

\sol

%%%%%%%%%%%%%%% (11)
\item Sean $A, B, C \in S$ eventos relacionados con el lanzamiento de un par de dados.
\begin{itemize}
    \item Si $\Pro(A|C) > \Pro(B|C)$ y $\Pro(A|C^c)>P(B|C^c)$
    Demuestra que: $Pro(A)>Pro(B)$ ó exhibe un contra-ejemplo definiendo eventos $A, B, C$ en donde
    dicha afirmación no se cumpla
    
    \item Si $\Pro(A|C) > \Pro(A|C^c)$ y $\Pro(B|C)>\Pro(B|C^c)$
    Demuestra que: $\Pro(AB|C) > \Pro(AB|C^c)$ ó exhibe un contra-ejemplo definiendo eventos 
    $A, B, C$ en donde dicha afirmación no se cumpla.

    \textit{Hint:} Sean $A, B, C \in S$ los eventos que definen si al lanzar un par de dados ocurre
    \begin{itemize}
        \item $C =$ {La suma de ambos dados es igual a 10}
        \item $A =$ {El Primer Dado cayo 6}
        \item $B =$ {El Segundo Dado cayo 6}
    \end{itemize}

\end{itemize}

%%%%%%%%%%%%%%% (12)
\item Jenny tiene ciertos criterios para elegir a su siguiente Novio. Ella tiene $n$ pretendientes
y compara entre si a cualquier par de chicos con los que sale y los clasifica. Ella puede decidir
tener una ``Cita'' con algún pretendiente en cualquier instante al azar. Cuando ella conoce bien a
algún pretendiente puede aceptar ser su novia ó bien lo rechaza (No sin antes...)

Si ella acepta ser Novia de algún pretendiente ella no intentara a los pretendientes con los que
aun no ha tenido una cita (En la vida real ... Whatsapp, Facebook, SNAPCHAT, etc. ¿Verdad?). Si ella
ha rechazado a algún pretendiente ella nunca mas lo podría considerar como opción para ser su novio.
La chica en el proceso de la búsqueda de su novio no elegirá a un pretendiente que comparado con los
chicos que anteriormente ha salido considere sea "peor". El objetivo de ella es maximizar la
probabilidad de elegir al ``mejor'' pretendiente para ser su novio. Y para ello sigue la siguiente
estrategia:

Para alguna $m$, $0 \leq m < n$ ella descarta a los primeros $m$ pretendientes. Ella acepta salir
con ellos y después de conocerlos los descarta sin importarle cual de ellos es mejor. Entonces
tomara como Novio, al siguiente pretendiente que considere mejor que los chicos a los que descarto.
En términos de $n$ ¿Cual es el valor de m que maximiza la probabilidad de que la chica elija al
mejor pretendiente ?

\textit{Hint: } Sean $E_m \in S$, $\{ B_i \}_{i=1}^{n} \subseteq S$ los eventos que definen
\begin{itemize}
    \item $E_m =$ {Jenny elige al mejor pretendiente}
    \item $B_i =$ {El mejor pretendiente es la i-esima cita de Jenny}
\end{itemize}
Por la ley de la Probabilidad Total se sigue que:
\[
    \Pro(E_m) = \sum_{i=1}^{\infty} \Pro(E_m | B_i) \Pro(Bi)
        \quad \text{donde } \Pro(B_i) = \frac{1}{n}
\]
Es evidente que $\Pro(E_m|B_i) = 0$ si $1 \leq i \leq m$, ahora si $i > m$ el i-esimo pretendiente
es el mejor entonces Jenny lo elegirá a el si y solo si el mejor de los anteriores esta dentro de
los primeros $m$ pretendientes que inicialmente descarto. De modo que
\[
    \Pro(E_m|B_i)= \frac{m}{i-1} \quad \text{si } i>m
\]
Ademas utiliza el hecho que
\[
    \sum_{i=m+1}^{n} \frac{1}{i+1} \approx \int_{m}^{n} \frac{1}{x}dx
\]
Ahora obtén el valor aproximado de $m$ para el cual $\Pro(E_m)$ es máxima.

%%%%%%%%%%%%%%% (13)
\item Una mujer llamada Paulina tiene 2 pretendientes: Ernesto y Luis. Ella esta decidida a ser
novia de alguno de ellos. Ernesto tiene una probabilidad de 0.7 de ser el elegido y Luis una de 0.3.
Si Ernesto es novio de Paulina hay una probabilidad de 0.4 de que terminen en matrimonio y si es
Luis, una de 0.3. Si despues de unos años hay reportes de que Paulina se caso con alguno de los
ellos. ¿Cual es la probabilidad de que haya sido con Luis?

%%%%%%%%%%%%%%% (12)
\item Investiga y resuelve el problema de Regla de la Sucesión de Laplace.

\textit{Hint:} Ver[Ross] Ejemplo 5e, Capitulo 3

\sol

\end{enumerate}
%%%%%%%%%%%%%%%%%%%%%%%%%%%%%%%%%%%%%%%%%%%%%%%%%%%%%%%%%%%%%%%%%%%%%%%%%%%%%%%%%%%%%%%%%


%%%%%%%%%%%%%%%%%%%%%%%%%%%%%%%%%%%%%%%%%%%%%%%%%%%%%%%%%%%%%%%%%%%%%%%%%%%%%%%%%%%%%%%%%

%%%%%%%%%%%%%%%%%%%%%%%%%%%%%%%%%%%%%%%%%%%%%%%%%%%%%%%%%%%%%%%%%%%%%%%%%%%%%%%%%%%%%%%%%

\end{document}